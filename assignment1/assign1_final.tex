% Options for packages loaded elsewhere
\PassOptionsToPackage{unicode}{hyperref}
\PassOptionsToPackage{hyphens}{url}
%
\documentclass[
]{article}
\usepackage{amsmath,amssymb}
\usepackage{iftex}
\ifPDFTeX
  \usepackage[T1]{fontenc}
  \usepackage[utf8]{inputenc}
  \usepackage{textcomp} % provide euro and other symbols
\else % if luatex or xetex
  \usepackage{unicode-math} % this also loads fontspec
  \defaultfontfeatures{Scale=MatchLowercase}
  \defaultfontfeatures[\rmfamily]{Ligatures=TeX,Scale=1}
\fi
\usepackage{lmodern}
\ifPDFTeX\else
  % xetex/luatex font selection
\fi
% Use upquote if available, for straight quotes in verbatim environments
\IfFileExists{upquote.sty}{\usepackage{upquote}}{}
\IfFileExists{microtype.sty}{% use microtype if available
  \usepackage[]{microtype}
  \UseMicrotypeSet[protrusion]{basicmath} % disable protrusion for tt fonts
}{}
\makeatletter
\@ifundefined{KOMAClassName}{% if non-KOMA class
  \IfFileExists{parskip.sty}{%
    \usepackage{parskip}
  }{% else
    \setlength{\parindent}{0pt}
    \setlength{\parskip}{6pt plus 2pt minus 1pt}}
}{% if KOMA class
  \KOMAoptions{parskip=half}}
\makeatother
\usepackage{xcolor}
\usepackage[margin=1in]{geometry}
\usepackage{color}
\usepackage{fancyvrb}
\newcommand{\VerbBar}{|}
\newcommand{\VERB}{\Verb[commandchars=\\\{\}]}
\DefineVerbatimEnvironment{Highlighting}{Verbatim}{commandchars=\\\{\}}
% Add ',fontsize=\small' for more characters per line
\usepackage{framed}
\definecolor{shadecolor}{RGB}{248,248,248}
\newenvironment{Shaded}{\begin{snugshade}}{\end{snugshade}}
\newcommand{\AlertTok}[1]{\textcolor[rgb]{0.94,0.16,0.16}{#1}}
\newcommand{\AnnotationTok}[1]{\textcolor[rgb]{0.56,0.35,0.01}{\textbf{\textit{#1}}}}
\newcommand{\AttributeTok}[1]{\textcolor[rgb]{0.13,0.29,0.53}{#1}}
\newcommand{\BaseNTok}[1]{\textcolor[rgb]{0.00,0.00,0.81}{#1}}
\newcommand{\BuiltInTok}[1]{#1}
\newcommand{\CharTok}[1]{\textcolor[rgb]{0.31,0.60,0.02}{#1}}
\newcommand{\CommentTok}[1]{\textcolor[rgb]{0.56,0.35,0.01}{\textit{#1}}}
\newcommand{\CommentVarTok}[1]{\textcolor[rgb]{0.56,0.35,0.01}{\textbf{\textit{#1}}}}
\newcommand{\ConstantTok}[1]{\textcolor[rgb]{0.56,0.35,0.01}{#1}}
\newcommand{\ControlFlowTok}[1]{\textcolor[rgb]{0.13,0.29,0.53}{\textbf{#1}}}
\newcommand{\DataTypeTok}[1]{\textcolor[rgb]{0.13,0.29,0.53}{#1}}
\newcommand{\DecValTok}[1]{\textcolor[rgb]{0.00,0.00,0.81}{#1}}
\newcommand{\DocumentationTok}[1]{\textcolor[rgb]{0.56,0.35,0.01}{\textbf{\textit{#1}}}}
\newcommand{\ErrorTok}[1]{\textcolor[rgb]{0.64,0.00,0.00}{\textbf{#1}}}
\newcommand{\ExtensionTok}[1]{#1}
\newcommand{\FloatTok}[1]{\textcolor[rgb]{0.00,0.00,0.81}{#1}}
\newcommand{\FunctionTok}[1]{\textcolor[rgb]{0.13,0.29,0.53}{\textbf{#1}}}
\newcommand{\ImportTok}[1]{#1}
\newcommand{\InformationTok}[1]{\textcolor[rgb]{0.56,0.35,0.01}{\textbf{\textit{#1}}}}
\newcommand{\KeywordTok}[1]{\textcolor[rgb]{0.13,0.29,0.53}{\textbf{#1}}}
\newcommand{\NormalTok}[1]{#1}
\newcommand{\OperatorTok}[1]{\textcolor[rgb]{0.81,0.36,0.00}{\textbf{#1}}}
\newcommand{\OtherTok}[1]{\textcolor[rgb]{0.56,0.35,0.01}{#1}}
\newcommand{\PreprocessorTok}[1]{\textcolor[rgb]{0.56,0.35,0.01}{\textit{#1}}}
\newcommand{\RegionMarkerTok}[1]{#1}
\newcommand{\SpecialCharTok}[1]{\textcolor[rgb]{0.81,0.36,0.00}{\textbf{#1}}}
\newcommand{\SpecialStringTok}[1]{\textcolor[rgb]{0.31,0.60,0.02}{#1}}
\newcommand{\StringTok}[1]{\textcolor[rgb]{0.31,0.60,0.02}{#1}}
\newcommand{\VariableTok}[1]{\textcolor[rgb]{0.00,0.00,0.00}{#1}}
\newcommand{\VerbatimStringTok}[1]{\textcolor[rgb]{0.31,0.60,0.02}{#1}}
\newcommand{\WarningTok}[1]{\textcolor[rgb]{0.56,0.35,0.01}{\textbf{\textit{#1}}}}
\usepackage{longtable,booktabs,array}
\usepackage{calc} % for calculating minipage widths
% Correct order of tables after \paragraph or \subparagraph
\usepackage{etoolbox}
\makeatletter
\patchcmd\longtable{\par}{\if@noskipsec\mbox{}\fi\par}{}{}
\makeatother
% Allow footnotes in longtable head/foot
\IfFileExists{footnotehyper.sty}{\usepackage{footnotehyper}}{\usepackage{footnote}}
\makesavenoteenv{longtable}
\usepackage{graphicx}
\makeatletter
\def\maxwidth{\ifdim\Gin@nat@width>\linewidth\linewidth\else\Gin@nat@width\fi}
\def\maxheight{\ifdim\Gin@nat@height>\textheight\textheight\else\Gin@nat@height\fi}
\makeatother
% Scale images if necessary, so that they will not overflow the page
% margins by default, and it is still possible to overwrite the defaults
% using explicit options in \includegraphics[width, height, ...]{}
\setkeys{Gin}{width=\maxwidth,height=\maxheight,keepaspectratio}
% Set default figure placement to htbp
\makeatletter
\def\fps@figure{htbp}
\makeatother
\setlength{\emergencystretch}{3em} % prevent overfull lines
\providecommand{\tightlist}{%
  \setlength{\itemsep}{0pt}\setlength{\parskip}{0pt}}
\setcounter{secnumdepth}{-\maxdimen} % remove section numbering
\ifLuaTeX
  \usepackage{selnolig}  % disable illegal ligatures
\fi
\usepackage{bookmark}
\IfFileExists{xurl.sty}{\usepackage{xurl}}{} % add URL line breaks if available
\urlstyle{same}
\hypersetup{
  pdftitle={Experimental Design and Data Analysis - Assignment 1},
  pdfauthor={Group 5 - Ivana Malčić, Xuening Tang, Xiaoxuan Zhang},
  hidelinks,
  pdfcreator={LaTeX via pandoc}}

\title{Experimental Design and Data Analysis - Assignment 1}
\author{Group 5 - Ivana Malčić, Xuening Tang, Xiaoxuan Zhang}
\date{2025-02-23}

\begin{document}
\maketitle

\subsection{Exercise 1: Cholesterol}\label{exercise-1-cholesterol}

\textbf{a)} In this first section, both normality and variable
correlation are explored using relevant plots and metrics. Firstly, the
bell-like shape of the histograms indicates that the data is normally
distributed.

\begin{Shaded}
\begin{Highlighting}[]
\NormalTok{cholesterol\_data }\OtherTok{\textless{}{-}} \FunctionTok{read.delim}\NormalTok{(}\StringTok{"cholesterol.txt"}\NormalTok{, }\AttributeTok{sep =} \StringTok{" "}\NormalTok{)}
\FunctionTok{par}\NormalTok{(}\AttributeTok{mfrow =} \FunctionTok{c}\NormalTok{(}\DecValTok{1}\NormalTok{,}\DecValTok{2}\NormalTok{))}

\CommentTok{\#histograms for \textquotesingle{}Before\textquotesingle{} and \textquotesingle{}After8weeks\textquotesingle{}}
\FunctionTok{hist}\NormalTok{(cholesterol\_data}\SpecialCharTok{$}\NormalTok{Before, }
     \AttributeTok{main =} \StringTok{"Histogram: Before"}\NormalTok{, }
     \AttributeTok{xlab =} \StringTok{"Cholesterol (Before)"}\NormalTok{, }
     \AttributeTok{col =} \StringTok{"lightblue"}\NormalTok{, }
     \AttributeTok{border =} \StringTok{"black"}\NormalTok{)}

\FunctionTok{hist}\NormalTok{(cholesterol\_data}\SpecialCharTok{$}\NormalTok{After8weeks, }
     \AttributeTok{main =} \StringTok{"Histogram: After8weeks"}\NormalTok{, }
     \AttributeTok{xlab =} \StringTok{"Cholesterol (After 8 weeks)"}\NormalTok{, }
     \AttributeTok{col =} \StringTok{"lightcoral"}\NormalTok{, }
     \AttributeTok{border =} \StringTok{"black"}\NormalTok{)}
\end{Highlighting}
\end{Shaded}

\includegraphics{assign1_final_files/figure-latex/unnamed-chunk-1-1.pdf}

The previous finding is further confirmed by the following QQ-plots
where the data points seem relatively close to the reference line, again
signaling normality.

\begin{Shaded}
\begin{Highlighting}[]
\FunctionTok{par}\NormalTok{(}\AttributeTok{mfrow =} \FunctionTok{c}\NormalTok{(}\DecValTok{1}\NormalTok{,}\DecValTok{2}\NormalTok{))}
\CommentTok{\#QQ{-}plots for \textquotesingle{}Before\textquotesingle{}}
\FunctionTok{qqnorm}\NormalTok{(cholesterol\_data}\SpecialCharTok{$}\NormalTok{Before, }\AttributeTok{main =} \StringTok{"QQ Plot: Before"}\NormalTok{)}
\FunctionTok{qqline}\NormalTok{(cholesterol\_data}\SpecialCharTok{$}\NormalTok{Before, }\AttributeTok{col =} \StringTok{"red"}\NormalTok{)}

\CommentTok{\#QQ{-}plots for \textquotesingle{}After8weeks\textquotesingle{}}
\FunctionTok{qqnorm}\NormalTok{(cholesterol\_data}\SpecialCharTok{$}\NormalTok{After8weeks, }\AttributeTok{main =} \StringTok{"QQ Plot: After8weeks"}\NormalTok{)}
\FunctionTok{qqline}\NormalTok{(cholesterol\_data}\SpecialCharTok{$}\NormalTok{After8weeks, }\AttributeTok{col =} \StringTok{"red"}\NormalTok{)}
\end{Highlighting}
\end{Shaded}

\begin{center}\includegraphics{assign1_final_files/figure-latex/unnamed-chunk-2-1} \end{center}

Additional data exploration gives us further insight; the close mean and
median signify symetric distribution, a feature which is also a common
attribute of normality. Moreover, the skewness for both variables tells
us that the left tail is slightly longer (distribution skewed to the
left). Finally, kurtosis of 2.5 and 2.27 indicates a peaked distribution
with less outliers and a more or less uniform distribution.

\begin{longtable}[]{@{}ccccc@{}}
\caption{Descriptive Statistics for Cholesterol Levels}\tabularnewline
\toprule\noalign{}
Variable & Mean & Median & Skewness & Kurtosis \\
\midrule\noalign{}
\endfirsthead
\toprule\noalign{}
Variable & Mean & Median & Skewness & Kurtosis \\
\midrule\noalign{}
\endhead
\bottomrule\noalign{}
\endlastfoot
Before & 6.41 & 6.50 & -0.28 & 2.50 \\
After8weeks & 5.78 & 5.73 & -0.17 & 2.27 \\
\end{longtable}

After normality assesment, we turn to look at whether the two variables
are correlated. For this we first utilize a simple scatterplot shown
below which exhibits strong positive correlation visible by the densly
clustered data points around the rising regression line.

\begin{center}\includegraphics{assign1_final_files/figure-latex/unnamed-chunk-4-1} \end{center}

Then, Pearson´s test is employed - the correlation coefficient of 0.991
indicates a strong and positive linear relationship between the two
variables. Furthermore, the small p-value (\textless0.001) suggests this
relationship is statistically significant, and therefore we can reject
the null hypothesis of no correlation.

\begin{longtable}[]{@{}ll@{}}
\caption{Pearson Correlation Test Results}\tabularnewline
\toprule\noalign{}
Statistic & Value \\
\midrule\noalign{}
\endfirsthead
\toprule\noalign{}
Statistic & Value \\
\midrule\noalign{}
\endhead
\bottomrule\noalign{}
\endlastfoot
Correlation Coefficient & 0.991 \\
P-value & 0 \\
Confidence Interval & 0.975 to 0.997 \\
\end{longtable}

\textbf{b)} Now, our goal is to establish whether the low-fat margarine
diet had any effect on cholesterol by utilizing 2 relevant test metrics.
Since our data is paired, we first utilize a paired t-test. The large
t-statistic and small p-value (p \textless{} 0.001) provide strong
evidence against the null hypothesis of no difference. Additionally, the
confidence interval suggests that the mean cholesterol level after 8
weekes lies somewhere between 0.54 and 0.718 with 95\% confidence.

\begin{longtable}[]{@{}ll@{}}
\caption{Paired t-Test Results}\tabularnewline
\toprule\noalign{}
Statistic & Value \\
\midrule\noalign{}
\endfirsthead
\toprule\noalign{}
Statistic & Value \\
\midrule\noalign{}
\endhead
\bottomrule\noalign{}
\endlastfoot
t-statistic & 14.946 \\
Degrees of Freedom & 17 \\
P-value & 0 \\
Confidence Interval & 0.54 to 0.718 \\
\end{longtable}

Since our data are paired and normally distributed, the Mann-Whitney U
test is not applicable in this scenario. However, we can apply the
permutation test which is useful because it works well with small data
volumes. The following permutation table reveals a similar trend as
previously discussed with a statistically significant (p\textless0.001)
average decrease in cholesterol levels by 0.629 units after the 8 week
intervention.

\begin{longtable}[]{@{}ll@{}}
\caption{Permutation Test Results}\tabularnewline
\toprule\noalign{}
Statistic & Value \\
\midrule\noalign{}
\endfirsthead
\toprule\noalign{}
Statistic & Value \\
\midrule\noalign{}
\endhead
\bottomrule\noalign{}
\endlastfoot
Observed Mean Difference & 0.629 \\
Permutation Test P-value & 0.000 \\
\end{longtable}

\textbf{c)} Next, we are constructing a 97\% CI and 97\% bootstrapped
CI, as opposed to our previously used 95\% CI. As visible from
\emph{Table 5}, we can be 97\% confident our true population parameter
is encompased between the ranges of {[}5.16, 6.39{]} for normal CI and
{[}5.23, 6.32{]} for the bootstrapped CI.

\begin{longtable}[]{@{}llrr@{}}
\caption{97\% Confidence Intervals for Mean}\tabularnewline
\toprule\noalign{}
& Method & Lower\_Bound & Upper\_Bound \\
\midrule\noalign{}
\endfirsthead
\toprule\noalign{}
& Method & Lower\_Bound & Upper\_Bound \\
\midrule\noalign{}
\endhead
\bottomrule\noalign{}
\endlastfoot
& Normality (t-distribution) & 5.16 & 6.39 \\
1.5\% & Bootstrap & 5.23 & 6.32 \\
\end{longtable}

\textbf{d)} Additionally, we use bootstrapping to come up with a 97\%
confidence interval for the maximum statistic for various candidate
values of θ, helping us reject or not reject the hypothesis that the
data follow a Uniform{[}3,θ{]} distribution. \emph{Table 6} provides us
with plausible candidate values for which we cannot reject the Null
hypothesis. Kolmogorov-Smirnov test can also be applied in this case to
test whether the data follows a uniform distribution.

\begin{longtable}[]{@{}ccc@{}}
\caption{Non-Rejected Theta Values}\tabularnewline
\toprule\noalign{}
Theta & Lower Bound & Upper Bound \\
\midrule\noalign{}
\endfirsthead
\toprule\noalign{}
Theta & Lower Bound & Upper Bound \\
\midrule\noalign{}
\endhead
\bottomrule\noalign{}
\endlastfoot
7.7 & 6.72 & 7.70 \\
7.8 & 6.81 & 7.80 \\
7.9 & 6.90 & 7.90 \\
8.0 & 6.98 & 8.00 \\
8.1 & 7.07 & 8.10 \\
8.2 & 7.13 & 8.20 \\
8.3 & 7.20 & 8.30 \\
8.4 & 7.29 & 8.39 \\
8.5 & 7.36 & 8.50 \\
8.6 & 7.45 & 8.60 \\
8.7 & 7.52 & 8.70 \\
8.8 & 7.57 & 8.79 \\
8.9 & 7.68 & 8.90 \\
\end{longtable}

Kolmogorov-Smirnov test can also be applied in this case to test whether
the data follows a uniform distribution.

\begin{longtable}[]{@{}cc@{}}
\caption{Theta Values with Non-Rejected KS Test}\tabularnewline
\toprule\noalign{}
Theta & P-Value \\
\midrule\noalign{}
\endfirsthead
\toprule\noalign{}
Theta & P-Value \\
\midrule\noalign{}
\endhead
\bottomrule\noalign{}
\endlastfoot
7.0 & 0.038 \\
7.1 & 0.053 \\
7.2 & 0.071 \\
7.3 & 0.092 \\
7.4 & 0.117 \\
7.5 & 0.146 \\
7.6 & 0.179 \\
7.7 & 0.216 \\
7.8 & 0.256 \\
7.9 & 0.299 \\
8.0 & 0.345 \\
8.1 & 0.394 \\
8.2 & 0.444 \\
8.3 & 0.495 \\
8.4 & 0.509 \\
8.5 & 0.419 \\
8.6 & 0.342 \\
8.7 & 0.277 \\
8.8 & 0.223 \\
8.9 & 0.179 \\
9.0 & 0.143 \\
9.1 & 0.114 \\
9.2 & 0.091 \\
9.3 & 0.072 \\
9.4 & 0.057 \\
9.5 & 0.045 \\
9.6 & 0.036 \\
\end{longtable}

\textbf{e)} Finally, we are testing the following Null hypothesis:
\emph{Null hypothesis (}\(H_0\)): The median cholesterol level after 8
weeks is 6. With the results presented below we can conclude there is
not enough statistical evidence to conclude that the median cholesterol
level after 8 weeks is less than 6. While 61.1\% of the sample is below
6, this deviation could easily be due to random variation given the
sample size (p\textgreater0.1).

\begin{longtable}[]{@{}ll@{}}
\caption{Median Test Results (H₀: median = 6)}\tabularnewline
\toprule\noalign{}
Statistic & Value \\
\midrule\noalign{}
\endfirsthead
\toprule\noalign{}
Statistic & Value \\
\midrule\noalign{}
\endhead
\bottomrule\noalign{}
\endlastfoot
Sample Size & 18 \\
Number \textless{} 6 & 11 \\
Observed Proportion & 0.611 \\
p-value & 0.24 \\
95\% CI & 0.392 to 1 \\
\end{longtable}

Subsequently, our second Null hypothesis goes as following: \emph{Null
hypothesis (}\(H_0\)): the fraction of cholesterol levels below 4.5 is
at most 0.25. Similarly, we also cannot reject this hypothesis because
of the very high p-value (p\textgreater0.1) and a wide CI.

\begin{longtable}[]{@{}ll@{}}
\caption{Fraction Test Results (H₀: fraction below 4.5 is
25\%)}\tabularnewline
\toprule\noalign{}
Statistic & Value \\
\midrule\noalign{}
\endfirsthead
\toprule\noalign{}
Statistic & Value \\
\midrule\noalign{}
\endhead
\bottomrule\noalign{}
\endlastfoot
Sample Size & 18 \\
Number \textless{} 4.5 & 3 \\
Observed Proportion & 0.167 \\
p-value & 0.865 \\
95\% CI & 0.047 to 1.000 \\
\end{longtable}

\subsection{Exercise 2}\label{exercise-2}

\textbf{a)} To study the effect of \textbf{County} and \textbf{Related}
on the variable \emph{Crops}, we propose the following hypotheses:

\begin{itemize}
\tightlist
\item
  \(H_{AB}\): There is no interaction effect of \emph{County} and
  \emph{Related}.
\item
  \(H_A\): There is no main effect of factor \emph{County}.
\item
  \(H_B\): There is no main effect of factor \emph{Related}.
\end{itemize}

Before conducting the ANOVA test, we first plot two interaction plots to
see potential interaction effect. Based on the left plot we can see
counties 1 and 2 cross lines in the beginning signaling slight
interaction when not \emph{Related}, but this effect is diminished when
they are \emph{Related} as the disparity grows larger. In the right plot
both lines are parallel, meaning the effect of counties on crop yields
does not really depend on being related or not.

\includegraphics{assign1_final_files/figure-latex/unnamed-chunk-14-1.pdf}

As visible in the ANOVA table below, results show that there is no
interaction effect between \emph{Related} and \emph{County} on
\emph{Crops} and none of the p-values for \emph{County}, \emph{Crops}
and \emph{County:Related} are significant (p = 0.477; p = 0.527; p =
0.879).

\begin{longtable}[]{@{}ccccc@{}}
\caption{Two-Way ANOVA Results}\tabularnewline
\toprule\noalign{}
Term & Sum of Squares & Df & F Value & Pr(\textgreater F) \\
\midrule\noalign{}
\endfirsthead
\toprule\noalign{}
Term & Sum of Squares & Df & F Value & Pr(\textgreater F) \\
\midrule\noalign{}
\endhead
\bottomrule\noalign{}
\endlastfoot
county & 8.84e+06 & 2 & 0.764 & 0.477 \\
related & 2.38e+06 & 1 & 0.411 & 0.527 \\
county:related & 1.50e+06 & 2 & 0.129 & 0.879 \\
Residuals & 1.39e+08 & 24 & NA & NA \\
\end{longtable}

\begin{longtable}[]{@{}
  >{\centering\arraybackslash}p{(\columnwidth - 8\tabcolsep) * \real{0.2609}}
  >{\centering\arraybackslash}p{(\columnwidth - 8\tabcolsep) * \real{0.1449}}
  >{\centering\arraybackslash}p{(\columnwidth - 8\tabcolsep) * \real{0.1739}}
  >{\centering\arraybackslash}p{(\columnwidth - 8\tabcolsep) * \real{0.1304}}
  >{\centering\arraybackslash}p{(\columnwidth - 8\tabcolsep) * \real{0.2899}}@{}}
\caption{Linear Model Coefficients}\tabularnewline
\toprule\noalign{}
\begin{minipage}[b]{\linewidth}\centering
Term
\end{minipage} & \begin{minipage}[b]{\linewidth}\centering
Estimate
\end{minipage} & \begin{minipage}[b]{\linewidth}\centering
Std. Error
\end{minipage} & \begin{minipage}[b]{\linewidth}\centering
t Value
\end{minipage} & \begin{minipage}[b]{\linewidth}\centering
Pr(\textgreater\textbar t\textbar)
\end{minipage} \\
\midrule\noalign{}
\endfirsthead
\toprule\noalign{}
\begin{minipage}[b]{\linewidth}\centering
Term
\end{minipage} & \begin{minipage}[b]{\linewidth}\centering
Estimate
\end{minipage} & \begin{minipage}[b]{\linewidth}\centering
Std. Error
\end{minipage} & \begin{minipage}[b]{\linewidth}\centering
t Value
\end{minipage} & \begin{minipage}[b]{\linewidth}\centering
Pr(\textgreater\textbar t\textbar)
\end{minipage} \\
\midrule\noalign{}
\endhead
\bottomrule\noalign{}
\endlastfoot
(Intercept) & 6700 & 1076 & 6.230 & 0.000 \\
county2 & 93 & 1521 & 0.061 & 0.952 \\
county3 & 851 & 1521 & 0.560 & 0.581 \\
related1 & -362 & 1521 & -0.238 & 0.814 \\
county2:related1 & -821 & 2151 & -0.381 & 0.706 \\
county3:related1 & 217 & 2151 & 0.101 & 0.920 \\
\end{longtable}

Next, we remove the interaction and apply an additive model. The result
of the additive model shows that neither of the factors has a
significant main effect on \emph{Crops}. The p-values of 0.4518 and
0.5126 for \emph{County} and \emph{Related} respectively, are larger
than the 0.05 significance level so we fail to reject our hypotheses.

\begin{longtable}[]{@{}ccccc@{}}
\caption{ANOVA Results for Additive Model}\tabularnewline
\toprule\noalign{}
Term & Sum of Squares & Df & F Value & Pr(\textgreater F) \\
\midrule\noalign{}
\endfirsthead
\toprule\noalign{}
Term & Sum of Squares & Df & F Value & Pr(\textgreater F) \\
\midrule\noalign{}
\endhead
\bottomrule\noalign{}
\endlastfoot
county & 8.84e+06 & 2 & 0.819 & 0.452 \\
related & 2.38e+06 & 1 & 0.441 & 0.513 \\
Residuals & 1.40e+08 & 26 & NA & NA \\
\end{longtable}

\begin{longtable}[]{@{}ccccc@{}}
\caption{Summary of Coefficients for Additive Model}\tabularnewline
\toprule\noalign{}
Term & Estimate & Std. Error & t Value &
Pr(\textgreater\textbar t\textbar) \\
\midrule\noalign{}
\endfirsthead
\toprule\noalign{}
Term & Estimate & Std. Error & t Value &
Pr(\textgreater\textbar t\textbar) \\
\midrule\noalign{}
\endhead
\bottomrule\noalign{}
\endlastfoot
Intercept & 6801 & 848 & 8.017 & 0.000 \\
County 2 & -317 & 1039 & -0.305 & 0.762 \\
County 3 & 960 & 1039 & 0.924 & 0.364 \\
Related 1 & -563 & 848 & -0.664 & 0.513 \\
\end{longtable}

The two-way ANOVA model is:

\[
Y_{ijk} = \mu_{ij} + e_{ijk}
\]

where

\[
\mu_{ij} = \mu + \alpha_i + \beta_j + \gamma_{ij}
\]

\begin{itemize}
\tightlist
\item
  \(\mu\) is the overall mean
\item
  \(\alpha_i\) is the main effect of \emph{County} (\(i = 1, 2, 3\))
\item
  \(\beta_j\) is the main effect of \emph{Related} (\(j = 0, 1\))
\item
  \(\gamma_{ij}\) is the interaction effect, which is 0 due to no
  significant interaction.
\end{itemize}

Using the model \textbf{cropanov2}, the predicted \emph{Crops} for
County 3 with no relation (Related = 0) is:

\[
Crops = 6800.6 + 959.7 + 0 = 7760.3
\]

Thus, the predicted value is 7760.3.

\textbf{b)} Next, we add the variable \emph{Size} into the analysis with
the aim to find out how it influences the effect of \emph{Related} or
\emph{County} on \emph{Crops} in our model.

\textbf{\emph{(A)}} ANCOVA test: \emph{Size * County} First, by getting
a gimpse in distributions for different counties, we conclude they are
different. Thus, we need to confirm our observation through a two-way
ANCOVA model.

\includegraphics{assign1_final_files/figure-latex/unnamed-chunk-17-1.pdf}

\begin{longtable}[]{@{}ccccc@{}}
\caption{ANOVA Table for Size and County Interaction}\tabularnewline
\toprule\noalign{}
Term & Sum\_Sq & Df & F\_Value & Pr\_F \\
\midrule\noalign{}
\endfirsthead
\toprule\noalign{}
Term & Sum\_Sq & Df & F\_Value & Pr\_F \\
\midrule\noalign{}
\endhead
\bottomrule\noalign{}
\endlastfoot
size & 1.20e+08 & 1 & 138.673 & 0.000 \\
county & 7.67e+05 & 2 & 0.445 & 0.646 \\
size:county & 1.05e+07 & 2 & 6.085 & 0.007 \\
Residuals & 2.07e+07 & 24 & NA & NA \\
\end{longtable}

\begin{longtable}[]{@{}ccccc@{}}
\caption{Model Summary for Size and County Interaction}\tabularnewline
\toprule\noalign{}
Term & Estimate & Std\_Error & t\_Value & Pr\_t \\
\midrule\noalign{}
\endfirsthead
\toprule\noalign{}
Term & Estimate & Std\_Error & t\_Value & Pr\_t \\
\midrule\noalign{}
\endhead
\bottomrule\noalign{}
\endlastfoot
Intercept & 2386.54 & 913.22 & 2.61 & 0.015 \\
Size & 22.46 & 4.70 & 4.78 & 0.000 \\
County 2 & -4370.41 & 1413.44 & -3.09 & 0.005 \\
County 3 & -1340.39 & 1285.69 & -1.04 & 0.308 \\
Size × County 2 & 27.51 & 7.89 & 3.49 & 0.002 \\
Size × County 3 & 9.21 & 6.31 & 1.46 & 0.157 \\
\end{longtable}

Based on the result, there is a significant interaction effect between
\emph{Size} and \emph{County} on \emph{Crops} (p-value = 0.007). Summary
of the ANOVA model shows that the effect mainly lies on the combination
of \emph{size:county 2} (p-value = 0.002), while \emph{size:county 3} is
not significant (p-value = 0.157). Meanwhile, \emph{county 2} also has a
significant main effect under the influence of \emph{Size} (p-value =
0.005).

\textbf{\emph{(B)}} ANCOVA \emph{Size * Related} In the box plot, the
distribution of \emph{Size} does not differ much for different
\emph{Related} values. The interaction effect is also not significant
according to the result of the ANCOVA test (p=0.331). Therefore, we
\textbf{cannot reject the null hypothesis} in this case.

\includegraphics{assign1_final_files/figure-latex/unnamed-chunk-19-1.pdf}

\begin{longtable}[]{@{}ccccc@{}}
\caption{ANOVA Results for Model with Size and Related}\tabularnewline
\toprule\noalign{}
Term & Sum of Squares & Df & F Value & Pr(\textgreater F) \\
\midrule\noalign{}
\endfirsthead
\toprule\noalign{}
Term & Sum of Squares & Df & F Value & Pr(\textgreater F) \\
\midrule\noalign{}
\endhead
\bottomrule\noalign{}
\endlastfoot
size & 1.20e+08 & 1 & 105.528 & 0.00 \\
related & 1.38e+06 & 1 & 1.218 & 0.28 \\
size:related & 1.11e+06 & 1 & 0.983 & 0.33 \\
Residuals & 2.95e+07 & 26 & NA & NA \\
\end{longtable}

\begin{longtable}[]{@{}ccccc@{}}
\caption{Model Coefficients for Size and Related}\tabularnewline
\toprule\noalign{}
Term & Estimate & Std. Error & t Value &
Pr(\textgreater\textbar t\textbar) \\
\midrule\noalign{}
\endfirsthead
\toprule\noalign{}
Term & Estimate & Std. Error & t Value &
Pr(\textgreater\textbar t\textbar) \\
\midrule\noalign{}
\endhead
\bottomrule\noalign{}
\endlastfoot
(Intercept) & 1774.98 & 940.16 & 1.888 & 0.070 \\
size & 28.21 & 4.84 & 5.828 & 0.000 \\
related1 & -1583.66 & 1227.31 & -1.290 & 0.208 \\
size:related1 & 6.27 & 6.33 & 0.992 & 0.330 \\
\end{longtable}

\textbf{\emph{(C)}} Lastly, we conduct two ANCOVA tests without
interaction. We investigate the main effect of \emph{Related} under the
influence of \emph{Size (cropanov5)}, and the main effect of \emph{Size}
under the influence of \emph{Related (cropanov6).} Results show that
\emph{Related} does not have a significant main effect on \emph{Crops}
under the influence of \emph{Size}, but \emph{Size} has a significant
main effect under the influence of \emph{Related}.

\begin{longtable}[]{@{}ccccc@{}}
\caption{ANOVA Results for Model with Size and Related (Size
first)}\tabularnewline
\toprule\noalign{}
Term & Sum of Squares & Df & F Value & Pr(\textgreater F) \\
\midrule\noalign{}
\endfirsthead
\toprule\noalign{}
Term & Sum of Squares & Df & F Value & Pr(\textgreater F) \\
\midrule\noalign{}
\endhead
\bottomrule\noalign{}
\endlastfoot
size & 1.20e+08 & 1 & 105.59 & 0.000 \\
related & 1.38e+06 & 1 & 1.22 & 0.279 \\
Residuals & 3.06e+07 & 27 & NA & NA \\
\end{longtable}

\begin{longtable}[]{@{}ccccc@{}}
\caption{Model Coefficients for Size and Related (Size
first)}\tabularnewline
\toprule\noalign{}
Term & Estimate & Std. Error & t Value &
Pr(\textgreater\textbar t\textbar) \\
\midrule\noalign{}
\endfirsthead
\toprule\noalign{}
Term & Estimate & Std. Error & t Value &
Pr(\textgreater\textbar t\textbar) \\
\midrule\noalign{}
\endhead
\bottomrule\noalign{}
\endlastfoot
(Intercept) & 1092.8 & 640.63 & 1.71 & 0.100 \\
size & 31.9 & 3.12 & 10.23 & 0.000 \\
related1 & -429.3 & 388.79 & -1.10 & 0.279 \\
\end{longtable}

\begin{longtable}[]{@{}ccccc@{}}
\caption{ANOVA Results for Model with Related and Size (Related
first)}\tabularnewline
\toprule\noalign{}
Term & Sum of Squares & Df & F Value & Pr(\textgreater F) \\
\midrule\noalign{}
\endfirsthead
\toprule\noalign{}
Term & Sum of Squares & Df & F Value & Pr(\textgreater F) \\
\midrule\noalign{}
\endhead
\bottomrule\noalign{}
\endlastfoot
related & 2.38e+06 & 1 & 2.1 & 0.159 \\
size & 1.19e+08 & 1 & 104.7 & 0.000 \\
Residuals & 3.06e+07 & 27 & NA & NA \\
\end{longtable}

\begin{longtable}[]{@{}ccccc@{}}
\caption{Model Coefficients for Related and Size (Related
first)}\tabularnewline
\toprule\noalign{}
Term & Estimate & Std. Error & t Value &
Pr(\textgreater\textbar t\textbar) \\
\midrule\noalign{}
\endfirsthead
\toprule\noalign{}
Term & Estimate & Std. Error & t Value &
Pr(\textgreater\textbar t\textbar) \\
\midrule\noalign{}
\endhead
\bottomrule\noalign{}
\endlastfoot
(Intercept) & 1092.8 & 640.63 & 1.71 & 0.100 \\
related1 & -429.3 & 388.79 & -1.10 & 0.279 \\
size & 31.9 & 3.12 & 10.23 & 0.000 \\
\end{longtable}

\textbf{c)} Based on our findings in part (b), we now include all
factors (\emph{Related} and \emph{County}) and the exploratory variable
(\emph{Size}) together in the same model, and subsequently conduct a
full ANCOVA test. The full ANCOVA test also confirms what we found in
section B in general. There is a significant main effect of variable
\emph{Size (p =0.000)} and \emph{County 2 (p = 0.008)}. The interaction
effect of \emph{Size:County} is also significant \emph{(p = 0.012)}.
Different from section B, \emph{County} now also has a slightly
significant main effect \emph{(p = 0.016)}, when \emph{County, Related}
and \emph{Size} are all included into the model. Based on the summary,
we can derive several conclusions:

\begin{itemize}
\item
  There is a significant difference between crop values in these three
  counties. County 2 yields significantly fewer crops than County 1
  (Estimate = -4214.050).
\item
  Size has a significantly positive effect on crops, with a larger land
  results in more crops (Estimate = 22.704).
\item
  The positive effect of size is more prominent in County 2, as there is
  a significant interaction effect. They yield a higher crops value than
  Size: County1 (Estimate = 26.590).
\item
  Value of crops does not depend on the relation of landlord and tenant
  in all three counties, ,as the difference between different relations
  is not statistically significant.
\end{itemize}

\begin{longtable}[]{@{}ccccc@{}}
\caption{ANOVA Results for Model with County, Related, Size, and
County:Size Interaction}\tabularnewline
\toprule\noalign{}
Term & Sum of Squares & Df & F Value & Pr(\textgreater F) \\
\midrule\noalign{}
\endfirsthead
\toprule\noalign{}
Term & Sum of Squares & Df & F Value & Pr(\textgreater F) \\
\midrule\noalign{}
\endhead
\bottomrule\noalign{}
\endlastfoot
county & 8.84e+06 & 2 & 5.01 & 0.016 \\
related & 2.38e+06 & 1 & 2.70 & 0.114 \\
size & 1.10e+08 & 1 & 125.33 & 0.000 \\
county:size & 9.53e+06 & 2 & 5.40 & 0.012 \\
Residuals & 2.03e+07 & 23 & NA & NA \\
\end{longtable}

\begin{longtable}[]{@{}ccccc@{}}
\caption{Model Coefficients for County, Related, Size, and County:Size
Interaction}\tabularnewline
\toprule\noalign{}
Term & Estimate & Std. Error & t Value &
Pr(\textgreater\textbar t\textbar) \\
\midrule\noalign{}
\endfirsthead
\toprule\noalign{}
Term & Estimate & Std. Error & t Value &
Pr(\textgreater\textbar t\textbar) \\
\midrule\noalign{}
\endhead
\bottomrule\noalign{}
\endlastfoot
(Intercept) & 2461.01 & 929.76 & 2.647 & 0.014 \\
county2 & -4214.05 & 1447.24 & -2.912 & 0.008 \\
county3 & -1284.81 & 1302.58 & -0.986 & 0.334 \\
relatedyes & -239.10 & 347.92 & -0.687 & 0.499 \\
size & 22.70 & 4.76 & 4.764 & 0.000 \\
county2:size & 26.59 & 8.09 & 3.286 & 0.003 \\
county3:size & 8.92 & 6.40 & 1.394 & 0.177 \\
\end{longtable}

\textbf{d)} We will apply model \textbf{cropanov7} to predict crops for
a farm from County 2 of size 165, with related landlord and tenant. To
do this we first have to know the mathematical formula for a full ANCOVA
which is:

\[ Y_{ijk} = \mu_{ij} + e_{ijk} \]

where

\[
\mu_{ijk} = \mu + \alpha_i + \beta_j + \delta_k + \gamma_{ik}
\]

\(\mu\) is the overall mean

\(\alpha_i\) is the main effect of level i of the factor \emph{County},
i = 1,2,3

\(\beta_j\) is the main effect of level j of the factor \emph{Related},
j = 0,1

\(\delta_k\) is the main effect of the exploratory variable Size, k =
1\ldots n

\(\gamma_ik\) is the interaction effect of levels i, k of factor
\emph{County} and exploratory variable \emph{Size}.

According to this equation, the crops from County 2 of size 165, and
related landlord and tenant is:

\[
Crops = Intercept + County2 + Related1 + Size 165 + County2*Size165
\]

\[
= 2461.014-4214.050-239.099+22.704*165+26.590*165 = 6141.378
\]

So the final crops value is 6141.378

The error variance is given by:

\[
\hat{\sigma}^2 = \frac{\text{RSS}}{\text{df}}
\]

According to the summary of the \textbf{cropanov7,} we then have:\[
\hat{\sigma}^2 = \frac{20277325}{23} = 881623
\]

The final estimated error variance is therefore 881623.

\subsection{Exercise 3: Yield of peas}\label{exercise-3-yield-of-peas}

\textbf{a)} For this exercise, we first present the R-code for the
randomization process to distribute soil additives over plots in such a
way that each soil additive is received exactly by two plots within each
block.

\begin{Shaded}
\begin{Highlighting}[]
\FunctionTok{set.seed}\NormalTok{(}\DecValTok{123}\NormalTok{)}

\CommentTok{\#initial params}
\NormalTok{I }\OtherTok{\textless{}{-}} \DecValTok{6}  \CommentTok{\# blocks}
\NormalTok{J }\OtherTok{\textless{}{-}} \DecValTok{4}  \CommentTok{\# plots per block}

\CommentTok{\#initial data frame}
\NormalTok{randomized\_design }\OtherTok{\textless{}{-}} \FunctionTok{data.frame}\NormalTok{(}
  \AttributeTok{block =} \FunctionTok{rep}\NormalTok{(}\DecValTok{1}\SpecialCharTok{:}\NormalTok{I, }\AttributeTok{each =}\NormalTok{ J),}
  \AttributeTok{plot =} \FunctionTok{rep}\NormalTok{(}\DecValTok{1}\SpecialCharTok{:}\NormalTok{J, }\AttributeTok{times =}\NormalTok{ I)}
\NormalTok{)}

\CommentTok{\# for each block b, put (N, P, K) on each 2 plots randomly}
\ControlFlowTok{for}\NormalTok{ (b }\ControlFlowTok{in} \DecValTok{1}\SpecialCharTok{:}\NormalTok{I) \{}
\NormalTok{  plots }\OtherTok{\textless{}{-}} \FunctionTok{sample}\NormalTok{(}\DecValTok{1}\SpecialCharTok{:}\NormalTok{J, J, }\AttributeTok{replace =} \ConstantTok{FALSE}\NormalTok{)  }\CommentTok{\# randomly reorder plots in each block}
  
  \CommentTok{\# put N in the header 2 plots}
\NormalTok{  randomized\_design}\SpecialCharTok{$}\NormalTok{N[randomized\_design}\SpecialCharTok{$}\NormalTok{block }\SpecialCharTok{==}\NormalTok{ b] }\OtherTok{\textless{}{-}} 
    \FunctionTok{ifelse}\NormalTok{(plots }\SpecialCharTok{\%in\%}\NormalTok{ plots[}\DecValTok{1}\SpecialCharTok{:}\DecValTok{2}\NormalTok{], }\DecValTok{1}\NormalTok{, }\DecValTok{0}\NormalTok{)}
  
  \CommentTok{\# randomly put P in 2 plots}
\NormalTok{  randomized\_design}\SpecialCharTok{$}\NormalTok{P[randomized\_design}\SpecialCharTok{$}\NormalTok{block }\SpecialCharTok{==}\NormalTok{ b] }\OtherTok{\textless{}{-}} 
    \FunctionTok{ifelse}\NormalTok{(plots }\SpecialCharTok{\%in\%} \FunctionTok{sample}\NormalTok{(plots, }\DecValTok{2}\NormalTok{), }\DecValTok{1}\NormalTok{, }\DecValTok{0}\NormalTok{)}
  
  \CommentTok{\# randomly put K in 2 plots}
\NormalTok{  randomized\_design}\SpecialCharTok{$}\NormalTok{K[randomized\_design}\SpecialCharTok{$}\NormalTok{block }\SpecialCharTok{==}\NormalTok{ b] }\OtherTok{\textless{}{-}} 
    \FunctionTok{ifelse}\NormalTok{(plots }\SpecialCharTok{\%in\%} \FunctionTok{sample}\NormalTok{(plots, }\DecValTok{2}\NormalTok{), }\DecValTok{1}\NormalTok{, }\DecValTok{0}\NormalTok{)}
\NormalTok{\}}

\CommentTok{\# Optionally print the randomized design}
\CommentTok{\# print(randomized\_design)}
\end{Highlighting}
\end{Shaded}

\textbf{b)} Then, the following plot illustrates that the average yields
for soil treated by N are higher than for untreated soil. What's more,
each block and treatment tend to have a similar change. Meanwhile, we
have assigned treatments randomly to each soil within a block, which
reduces the variation and gets more precise results.

\begin{center}\includegraphics{assign1_final_files/figure-latex/unnamed-chunk-24-1} \end{center}

\textbf{c)} For this part of the report we conduct a full two-way ANOVA
with the response variable yield and the two factors; block and N. As
visible from the results tables, p\textgreater{} 0.05, meaning there is
no significant evidence of interaction effect.

\begin{longtable}[]{@{}cccccc@{}}
\caption{Two-Way ANOVA Results for Yield by Nitrogen and Block (With
Interaction)}\tabularnewline
\toprule\noalign{}
Term & Df & Sum Sq & Mean Sq & F Value & Pr(\textgreater F) \\
\midrule\noalign{}
\endfirsthead
\toprule\noalign{}
Term & Df & Sum Sq & Mean Sq & F Value & Pr(\textgreater F) \\
\midrule\noalign{}
\endhead
\bottomrule\noalign{}
\endlastfoot
N & 1 & 189.3 & 189.3 & 9.261 & 0.010 \\
block & 5 & 343.3 & 68.7 & 3.359 & 0.040 \\
N:block & 5 & 98.5 & 19.7 & 0.964 & 0.477 \\
Residuals & 12 & 245.3 & 20.4 & NA & NA \\
\end{longtable}

Now we turn to generating an additive model. Here in both cases
p\textless0.05, so both factors have a main effect. From the results
showed in table below, the p\_value of block is 0.007095 \textless{}
0.05, and the N is the first order in our model, so it makes sense to
include the block. The Fridman test cannot be applied in this situation
because each block has more than one same value N, meanwhile, the
treatments are not completely randomized.

\begin{longtable}[]{@{}cccccc@{}}
\caption{ANOVA Results for Yield by Nitrogen and Block (No
Interaction)}\tabularnewline
\toprule\noalign{}
Term & Df & Sum Sq & Mean Sq & F Value & Pr(\textgreater F) \\
\midrule\noalign{}
\endfirsthead
\toprule\noalign{}
Term & Df & Sum Sq & Mean Sq & F Value & Pr(\textgreater F) \\
\midrule\noalign{}
\endhead
\bottomrule\noalign{}
\endlastfoot
N & 1 & 189 & 189.3 & 9.36 & 0.007 \\
block & 5 & 343 & 68.7 & 3.40 & 0.026 \\
Residuals & 17 & 344 & 20.2 & NA & NA \\
\end{longtable}

\textbf{d)} We want to explore interaction terms now, so first we do
interactions with a pairwise model. From the results we see there is no
interaction effect for either of the three.

\begin{longtable}[]{@{}cccccc@{}}
\caption{ANOVA Results for Pairwise P Model (block*P + N +
K)}\tabularnewline
\toprule\noalign{}
Term & Df & Sum Sq & Mean Sq & F Value & Pr(\textgreater F) \\
\midrule\noalign{}
\endfirsthead
\toprule\noalign{}
Term & Df & Sum Sq & Mean Sq & F Value & Pr(\textgreater F) \\
\midrule\noalign{}
\endhead
\bottomrule\noalign{}
\endlastfoot
block & 5 & 343.3 & 68.7 & 4.068 & 0.028 \\
P & 1 & 8.4 & 8.4 & 0.498 & 0.497 \\
N & 1 & 189.3 & 189.3 & 11.214 & 0.007 \\
K & 1 & 95.2 & 95.2 & 5.640 & 0.039 \\
block:P & 5 & 71.4 & 14.3 & 0.846 & 0.547 \\
Residuals & 10 & 168.8 & 16.9 & NA & NA \\
\end{longtable}

\begin{longtable}[]{@{}cccccc@{}}
\caption{ANOVA Results for Pairwise K Model (block*K + P +
N)}\tabularnewline
\toprule\noalign{}
Term & Df & Sum Sq & Mean Sq & F Value & Pr(\textgreater F) \\
\midrule\noalign{}
\endfirsthead
\toprule\noalign{}
Term & Df & Sum Sq & Mean Sq & F Value & Pr(\textgreater F) \\
\midrule\noalign{}
\endhead
\bottomrule\noalign{}
\endlastfoot
block & 5 & 343.3 & 68.7 & 4.041 & 0.029 \\
K & 1 & 95.2 & 95.2 & 5.603 & 0.039 \\
P & 1 & 8.4 & 8.4 & 0.494 & 0.498 \\
N & 1 & 189.3 & 189.3 & 11.140 & 0.008 \\
block:K & 5 & 70.3 & 14.1 & 0.827 & 0.558 \\
Residuals & 10 & 169.9 & 17.0 & NA & NA \\
\end{longtable}

\begin{longtable}[]{@{}cccccc@{}}
\caption{ANOVA Results for Pairwise N Model (block*N + K +
P)}\tabularnewline
\toprule\noalign{}
Term & Df & Sum Sq & Mean Sq & F Value & Pr(\textgreater F) \\
\midrule\noalign{}
\endfirsthead
\toprule\noalign{}
Term & Df & Sum Sq & Mean Sq & F Value & Pr(\textgreater F) \\
\midrule\noalign{}
\endhead
\bottomrule\noalign{}
\endlastfoot
block & 5 & 343.3 & 68.7 & 4.847 & 0.016 \\
N & 1 & 189.3 & 189.3 & 13.361 & 0.004 \\
K & 1 & 95.2 & 95.2 & 6.720 & 0.027 \\
P & 1 & 8.4 & 8.4 & 0.593 & 0.459 \\
block:N & 5 & 98.5 & 19.7 & 1.391 & 0.307 \\
Residuals & 10 & 141.7 & 14.2 & NA & NA \\
\end{longtable}

Then, we turn to an additive model. Here we notice p\textless0.05 for N
and K, showing a main effect. However, for P p\textgreater0.05, so we
can conclude that there is no significant effect. \textbf{We conclude
that the best model is the additive ANOVA model.} Additive ANOVA model
provides an overall indication of the effects of each factor N, P, K and
block. From the p values we can tell additive model are better than the
pairwise model. Furthermore, the pairwise models only focus on the
interaction between two factors at a time, lacking the control for
others.

\begin{longtable}[]{@{}lccccc@{}}
\caption{ANOVA Results for Additive Model (Yield \textasciitilde{} block
+ N + P + K)}\tabularnewline
\toprule\noalign{}
Term & Df & Sum Sq & Mean Sq & F value & Pr(\textgreater F) \\
\midrule\noalign{}
\endfirsthead
\toprule\noalign{}
Term & Df & Sum Sq & Mean Sq & F value & Pr(\textgreater F) \\
\midrule\noalign{}
\endhead
\bottomrule\noalign{}
\endlastfoot
block & 5 & 343.3 & 68.7 & 4.288 & 0.013 \\
N & 1 & 189.3 & 189.3 & 11.821 & 0.004 \\
P & 1 & 8.4 & 8.4 & 0.525 & 0.480 \\
K & 1 & 95.2 & 95.2 & 5.946 & 0.028 \\
Residuals & 15 & 240.2 & 16.0 & NA & NA \\
\end{longtable}

\textbf{e)} From our resulting model in part d), we further investigate
how the involved factors influence yield. From the summary we can see
block 3 is the best in all blocks, and N1 is better than N0, which means
N treated is better, while P and K are prefered to be untreated. So, the
best combination is \textbf{(3, 1, 0, 0)} for (block, N, P, K), leading
to the largest yield.

\begin{longtable}[]{@{}lcccc@{}}
\caption{Model Coefficients for Yield \textasciitilde{} block + N + K +
P}\tabularnewline
\toprule\noalign{}
& Estimate & Std. Error & t Value &
Pr(\textgreater\textbar t\textbar) \\
\midrule\noalign{}
\endfirsthead
\toprule\noalign{}
& Estimate & Std. Error & t Value &
Pr(\textgreater\textbar t\textbar) \\
\midrule\noalign{}
\endhead
\bottomrule\noalign{}
\endlastfoot
(Intercept) & 53.80 & 2.45 & 21.955 & 0.000 \\
block2 & 3.42 & 2.83 & 1.210 & 0.245 \\
block3 & 6.75 & 2.83 & 2.386 & 0.031 \\
block4 & -3.90 & 2.83 & -1.378 & 0.188 \\
block5 & -3.50 & 2.83 & -1.237 & 0.235 \\
block6 & 2.33 & 2.83 & 0.822 & 0.424 \\
N1 & 5.62 & 1.63 & 3.438 & 0.004 \\
K1 & -3.98 & 1.63 & -2.438 & 0.028 \\
P1 & -1.18 & 1.63 & -0.724 & 0.480 \\
\end{longtable}

\textbf{f)} In conclusion, we want to perform a mixed effects analysis
for our model from task d), modeling the block variable as a random
effect. Then we compare our results to the results found by using the
fixed effects model. These are our findings: - AIC: The fixed effects
model (AIC = 143.39) is lower than the mixed effects model (AIC =
151.03), suggesting better model fit. - BIC: The fixed effects model
(BIC = 155.17) is also lower, reinforcing the AIC results. -
Log-likelihood: The fixed effects model has a higher log-likelihood
(-61.695 vs.~-69.514), meaning it fits the data better. - Chi-square
test: χ² = 15.639, p = 0.003544 (significant at p \textless{} 0.05),
indicating that treating block as a fixed effect is more appropriate.

The fixed effects model (lm(yield \textasciitilde{} block + N + P + K))
provides a better fit than the mixed effects model.

\begin{longtable}[]{@{}
  >{\raggedright\arraybackslash}p{(\columnwidth - 16\tabcolsep) * \real{0.5956}}
  >{\centering\arraybackslash}p{(\columnwidth - 16\tabcolsep) * \real{0.0515}}
  >{\centering\arraybackslash}p{(\columnwidth - 16\tabcolsep) * \real{0.0294}}
  >{\centering\arraybackslash}p{(\columnwidth - 16\tabcolsep) * \real{0.0368}}
  >{\centering\arraybackslash}p{(\columnwidth - 16\tabcolsep) * \real{0.0368}}
  >{\centering\arraybackslash}p{(\columnwidth - 16\tabcolsep) * \real{0.0588}}
  >{\centering\arraybackslash}p{(\columnwidth - 16\tabcolsep) * \real{0.0588}}
  >{\centering\arraybackslash}p{(\columnwidth - 16\tabcolsep) * \real{0.0662}}
  >{\centering\arraybackslash}p{(\columnwidth - 16\tabcolsep) * \real{0.0662}}@{}}
\caption{ANOVA Comparison of Mixed Model (with block as random effect)
and Fixed Model}\tabularnewline
\toprule\noalign{}
\begin{minipage}[b]{\linewidth}\raggedright
Call
\end{minipage} & \begin{minipage}[b]{\linewidth}\centering
Model
\end{minipage} & \begin{minipage}[b]{\linewidth}\centering
Df
\end{minipage} & \begin{minipage}[b]{\linewidth}\centering
AIC
\end{minipage} & \begin{minipage}[b]{\linewidth}\centering
BIC
\end{minipage} & \begin{minipage}[b]{\linewidth}\centering
LogLik
\end{minipage} & \begin{minipage}[b]{\linewidth}\centering
Test
\end{minipage} & \begin{minipage}[b]{\linewidth}\centering
L.Ratio
\end{minipage} & \begin{minipage}[b]{\linewidth}\centering
p-value
\end{minipage} \\
\midrule\noalign{}
\endfirsthead
\toprule\noalign{}
\begin{minipage}[b]{\linewidth}\raggedright
Call
\end{minipage} & \begin{minipage}[b]{\linewidth}\centering
Model
\end{minipage} & \begin{minipage}[b]{\linewidth}\centering
Df
\end{minipage} & \begin{minipage}[b]{\linewidth}\centering
AIC
\end{minipage} & \begin{minipage}[b]{\linewidth}\centering
BIC
\end{minipage} & \begin{minipage}[b]{\linewidth}\centering
LogLik
\end{minipage} & \begin{minipage}[b]{\linewidth}\centering
Test
\end{minipage} & \begin{minipage}[b]{\linewidth}\centering
L.Ratio
\end{minipage} & \begin{minipage}[b]{\linewidth}\centering
p-value
\end{minipage} \\
\midrule\noalign{}
\endhead
\bottomrule\noalign{}
\endlastfoot
lme.formula(fixed = yield \textasciitilde{} N + P + K, data = npk,
random = \textasciitilde1 \textbar{} block) & 1 & 6 & 140 & 146 & -64.0
& & NA & NA \\
lm(formula = yield \textasciitilde{} N + P + K + block, data = npk) & 2
& 10 & 118 & 125 & -48.9 & 1 vs 2 & 30.2 & 0 \\
\end{longtable}

\end{document}

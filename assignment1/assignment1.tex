% Options for packages loaded elsewhere
\PassOptionsToPackage{unicode}{hyperref}
\PassOptionsToPackage{hyphens}{url}
%
\documentclass[
]{article}
\usepackage{amsmath,amssymb}
\usepackage{iftex}
\ifPDFTeX
  \usepackage[T1]{fontenc}
  \usepackage[utf8]{inputenc}
  \usepackage{textcomp} % provide euro and other symbols
\else % if luatex or xetex
  \usepackage{unicode-math} % this also loads fontspec
  \defaultfontfeatures{Scale=MatchLowercase}
  \defaultfontfeatures[\rmfamily]{Ligatures=TeX,Scale=1}
\fi
\usepackage{lmodern}
\ifPDFTeX\else
  % xetex/luatex font selection
\fi
% Use upquote if available, for straight quotes in verbatim environments
\IfFileExists{upquote.sty}{\usepackage{upquote}}{}
\IfFileExists{microtype.sty}{% use microtype if available
  \usepackage[]{microtype}
  \UseMicrotypeSet[protrusion]{basicmath} % disable protrusion for tt fonts
}{}
\makeatletter
\@ifundefined{KOMAClassName}{% if non-KOMA class
  \IfFileExists{parskip.sty}{%
    \usepackage{parskip}
  }{% else
    \setlength{\parindent}{0pt}
    \setlength{\parskip}{6pt plus 2pt minus 1pt}}
}{% if KOMA class
  \KOMAoptions{parskip=half}}
\makeatother
\usepackage{xcolor}
\usepackage[margin=1in]{geometry}
\usepackage{color}
\usepackage{fancyvrb}
\newcommand{\VerbBar}{|}
\newcommand{\VERB}{\Verb[commandchars=\\\{\}]}
\DefineVerbatimEnvironment{Highlighting}{Verbatim}{commandchars=\\\{\}}
% Add ',fontsize=\small' for more characters per line
\usepackage{framed}
\definecolor{shadecolor}{RGB}{248,248,248}
\newenvironment{Shaded}{\begin{snugshade}}{\end{snugshade}}
\newcommand{\AlertTok}[1]{\textcolor[rgb]{0.94,0.16,0.16}{#1}}
\newcommand{\AnnotationTok}[1]{\textcolor[rgb]{0.56,0.35,0.01}{\textbf{\textit{#1}}}}
\newcommand{\AttributeTok}[1]{\textcolor[rgb]{0.13,0.29,0.53}{#1}}
\newcommand{\BaseNTok}[1]{\textcolor[rgb]{0.00,0.00,0.81}{#1}}
\newcommand{\BuiltInTok}[1]{#1}
\newcommand{\CharTok}[1]{\textcolor[rgb]{0.31,0.60,0.02}{#1}}
\newcommand{\CommentTok}[1]{\textcolor[rgb]{0.56,0.35,0.01}{\textit{#1}}}
\newcommand{\CommentVarTok}[1]{\textcolor[rgb]{0.56,0.35,0.01}{\textbf{\textit{#1}}}}
\newcommand{\ConstantTok}[1]{\textcolor[rgb]{0.56,0.35,0.01}{#1}}
\newcommand{\ControlFlowTok}[1]{\textcolor[rgb]{0.13,0.29,0.53}{\textbf{#1}}}
\newcommand{\DataTypeTok}[1]{\textcolor[rgb]{0.13,0.29,0.53}{#1}}
\newcommand{\DecValTok}[1]{\textcolor[rgb]{0.00,0.00,0.81}{#1}}
\newcommand{\DocumentationTok}[1]{\textcolor[rgb]{0.56,0.35,0.01}{\textbf{\textit{#1}}}}
\newcommand{\ErrorTok}[1]{\textcolor[rgb]{0.64,0.00,0.00}{\textbf{#1}}}
\newcommand{\ExtensionTok}[1]{#1}
\newcommand{\FloatTok}[1]{\textcolor[rgb]{0.00,0.00,0.81}{#1}}
\newcommand{\FunctionTok}[1]{\textcolor[rgb]{0.13,0.29,0.53}{\textbf{#1}}}
\newcommand{\ImportTok}[1]{#1}
\newcommand{\InformationTok}[1]{\textcolor[rgb]{0.56,0.35,0.01}{\textbf{\textit{#1}}}}
\newcommand{\KeywordTok}[1]{\textcolor[rgb]{0.13,0.29,0.53}{\textbf{#1}}}
\newcommand{\NormalTok}[1]{#1}
\newcommand{\OperatorTok}[1]{\textcolor[rgb]{0.81,0.36,0.00}{\textbf{#1}}}
\newcommand{\OtherTok}[1]{\textcolor[rgb]{0.56,0.35,0.01}{#1}}
\newcommand{\PreprocessorTok}[1]{\textcolor[rgb]{0.56,0.35,0.01}{\textit{#1}}}
\newcommand{\RegionMarkerTok}[1]{#1}
\newcommand{\SpecialCharTok}[1]{\textcolor[rgb]{0.81,0.36,0.00}{\textbf{#1}}}
\newcommand{\SpecialStringTok}[1]{\textcolor[rgb]{0.31,0.60,0.02}{#1}}
\newcommand{\StringTok}[1]{\textcolor[rgb]{0.31,0.60,0.02}{#1}}
\newcommand{\VariableTok}[1]{\textcolor[rgb]{0.00,0.00,0.00}{#1}}
\newcommand{\VerbatimStringTok}[1]{\textcolor[rgb]{0.31,0.60,0.02}{#1}}
\newcommand{\WarningTok}[1]{\textcolor[rgb]{0.56,0.35,0.01}{\textbf{\textit{#1}}}}
\usepackage{longtable,booktabs,array}
\usepackage{calc} % for calculating minipage widths
% Correct order of tables after \paragraph or \subparagraph
\usepackage{etoolbox}
\makeatletter
\patchcmd\longtable{\par}{\if@noskipsec\mbox{}\fi\par}{}{}
\makeatother
% Allow footnotes in longtable head/foot
\IfFileExists{footnotehyper.sty}{\usepackage{footnotehyper}}{\usepackage{footnote}}
\makesavenoteenv{longtable}
\usepackage{graphicx}
\makeatletter
\def\maxwidth{\ifdim\Gin@nat@width>\linewidth\linewidth\else\Gin@nat@width\fi}
\def\maxheight{\ifdim\Gin@nat@height>\textheight\textheight\else\Gin@nat@height\fi}
\makeatother
% Scale images if necessary, so that they will not overflow the page
% margins by default, and it is still possible to overwrite the defaults
% using explicit options in \includegraphics[width, height, ...]{}
\setkeys{Gin}{width=\maxwidth,height=\maxheight,keepaspectratio}
% Set default figure placement to htbp
\makeatletter
\def\fps@figure{htbp}
\makeatother
\setlength{\emergencystretch}{3em} % prevent overfull lines
\providecommand{\tightlist}{%
  \setlength{\itemsep}{0pt}\setlength{\parskip}{0pt}}
\setcounter{secnumdepth}{-\maxdimen} % remove section numbering
\usepackage{booktabs}
\usepackage{longtable}
\usepackage{array}
\usepackage{multirow}
\usepackage{wrapfig}
\usepackage{float}
\usepackage{colortbl}
\usepackage{pdflscape}
\usepackage{tabu}
\usepackage{threeparttable}
\usepackage{threeparttablex}
\usepackage[normalem]{ulem}
\usepackage{makecell}
\usepackage{xcolor}
\ifLuaTeX
  \usepackage{selnolig}  % disable illegal ligatures
\fi
\usepackage{bookmark}
\IfFileExists{xurl.sty}{\usepackage{xurl}}{} % add URL line breaks if available
\urlstyle{same}
\hypersetup{
  pdftitle={Experimental Design and Data Analysis - Assignment 1},
  pdfauthor={Group 5 - Ivana Malčić, Xuening Tang, Xiaoxuan Zhang},
  hidelinks,
  pdfcreator={LaTeX via pandoc}}

\title{Experimental Design and Data Analysis - Assignment 1}
\author{Group 5 - Ivana Malčić, Xuening Tang, Xiaoxuan Zhang}
\date{2025-02-23}

\begin{document}
\maketitle

\emph{In order not to be bothered with rounding the numbers, set
\texttt{options(digits=3)r\ options(digits=3)}.}

\subsection{Exercise 1: Cholesterol}\label{exercise-1-cholesterol}

\textbf{a)} In this first section, both normality and variable
correlation are explored using relevant plots and metrics. Firstly, the
bell-like shape of the histograms indicates that the data is normally
distributed.

\includegraphics{assignment1_files/figure-latex/unnamed-chunk-1-1.pdf}

The previous finding is further confirmed by the following QQ-plots
where the data points seem relatively close to the reference line, again
signaling normality.

\begin{center}\includegraphics{assignment1_files/figure-latex/unnamed-chunk-2-1} \end{center}

Additional data exploration gives us further insight; the close mean and
median signify symetric distribution, a feature which is also a common
attribute of normality. Moreover, the skewness for both variables tells
us that the left tail is slightly longer (distribution skewed to the
left). Finally, kurtosis of 2.5 and 2.27 indicates a peaked distribution
with less outliers and a more or less uniform distribution.

\begin{longtable}[]{@{}ccccc@{}}
\caption{Descriptive Statistics for Cholesterol Levels}\tabularnewline
\toprule\noalign{}
Variable & Mean & Median & Skewness & Kurtosis \\
\midrule\noalign{}
\endfirsthead
\toprule\noalign{}
Variable & Mean & Median & Skewness & Kurtosis \\
\midrule\noalign{}
\endhead
\bottomrule\noalign{}
\endlastfoot
Before & 6.41 & 6.50 & -0.28 & 2.50 \\
After8weeks & 5.78 & 5.73 & -0.17 & 2.27 \\
\end{longtable}

After normality assesment, we turn to look at whether the two variables
are correlated. For this we first utilize a simple scatterplot shown
below which exhibits strong positive correlation visible by the densly
clustered data points around the rising regression line.

\begin{center}\includegraphics{assignment1_files/figure-latex/unnamed-chunk-4-1} \end{center}

Then, Pearson´s test is employed - the correlation coefficient of 0.991
indicates a strong and positive linear relationship between the two
variables. Furthermore, the small p-value (\textless0.001) suggests this
relationship is statistically significant, and therefore we can reject
the null hypothesis of no correlation.

\begin{longtable}[]{@{}ll@{}}
\caption{Pearson Correlation Test Results}\tabularnewline
\toprule\noalign{}
Statistic & Value \\
\midrule\noalign{}
\endfirsthead
\toprule\noalign{}
Statistic & Value \\
\midrule\noalign{}
\endhead
\bottomrule\noalign{}
\endlastfoot
Correlation Coefficient & 0.991 \\
P-value & 0 \\
Confidence Interval & 0.975 to 0.997 \\
\end{longtable}

\textbf{b)} Now, our goal is to establish whether the low-fat margarine
diet had any effect on cholesterol by utilizing 2 relevant test metrics.
Since our data is paired, we first utilize a paired t-test. The large
t-statistic and small p-value (p \textless{} 0.001) provide strong
evidence against the null hypothesis of no difference. Additionally, the
confidence interval suggests that the mean cholesterol level after 8
weekes lies somewhere between 0.54 and 0.718 with 95\% confidence.

\begin{longtable}[]{@{}ll@{}}
\caption{Paired t-Test Results}\tabularnewline
\toprule\noalign{}
Statistic & Value \\
\midrule\noalign{}
\endfirsthead
\toprule\noalign{}
Statistic & Value \\
\midrule\noalign{}
\endhead
\bottomrule\noalign{}
\endlastfoot
t-statistic & 14.946 \\
Degrees of Freedom & 17 \\
P-value & 0 \\
Confidence Interval & 0.54 to 0.718 \\
\end{longtable}

Since our data are paired and normally distributed, the Mann-Whitney U
test is not applicable in this scenario. However, we can apply the
permutation test which is useful because it works well with small data
volumes. The following permutation table reveals a similar trend as
previously discussed with a statistically significant (p\textless0.001)
average decrease in cholesterol levels by 0.629 units after the 8 week
intervention.

\begin{longtable}[]{@{}ll@{}}
\caption{Permutation Test Results}\tabularnewline
\toprule\noalign{}
Statistic & Value \\
\midrule\noalign{}
\endfirsthead
\toprule\noalign{}
Statistic & Value \\
\midrule\noalign{}
\endhead
\bottomrule\noalign{}
\endlastfoot
Observed Mean Difference & 0.629 \\
Permutation Test P-value & 0.000 \\
\end{longtable}

\textbf{c)} Next, we are constructing a 97\% CI and 97\% bootstrapped
CI, as opposed to our previously used 95\% CI. As visible from
\emph{Table 5}, we can be 97\% confident our true population parameter
is encompased between the ranges of {[}5.16, 6.39{]} for normal CI and
{[}5.23, 6.32{]} for the bootstrapped CI.

\begin{longtable}[]{@{}llrr@{}}
\caption{97\% Confidence Intervals for Mean}\tabularnewline
\toprule\noalign{}
& Method & Lower\_Bound & Upper\_Bound \\
\midrule\noalign{}
\endfirsthead
\toprule\noalign{}
& Method & Lower\_Bound & Upper\_Bound \\
\midrule\noalign{}
\endhead
\bottomrule\noalign{}
\endlastfoot
& Normality (t-distribution) & 5.16 & 6.39 \\
1.5\% & Bootstrap & 5.23 & 6.32 \\
\end{longtable}

\textbf{d)} Additionally, we use bootstrapping to come up with a 97\%
confidence interval for the maximum statistic for various candidate
values of θ, helping us reject or not reject the hypothesis that the
data follow a Uniform{[}3,θ{]} distribution. \emph{Table 6} provides us
with plausible candidate values for which we cannot reject the Null
hypothesis. Kolmogorov-Smirnov test can also be applied in this case to
test whether the data follows a uniform distribution.

\begin{longtable}[]{@{}ccc@{}}
\caption{Non-Rejected Theta Values}\tabularnewline
\toprule\noalign{}
Theta & Lower Bound & Upper Bound \\
\midrule\noalign{}
\endfirsthead
\toprule\noalign{}
Theta & Lower Bound & Upper Bound \\
\midrule\noalign{}
\endhead
\bottomrule\noalign{}
\endlastfoot
7.7 & 6.72 & 7.70 \\
7.8 & 6.81 & 7.80 \\
7.9 & 6.90 & 7.90 \\
8.0 & 6.98 & 8.00 \\
8.1 & 7.07 & 8.10 \\
8.2 & 7.13 & 8.20 \\
8.3 & 7.20 & 8.30 \\
8.4 & 7.29 & 8.39 \\
8.5 & 7.36 & 8.50 \\
8.6 & 7.45 & 8.60 \\
8.7 & 7.52 & 8.70 \\
8.8 & 7.57 & 8.79 \\
8.9 & 7.68 & 8.90 \\
\end{longtable}

Kolmogorov-Smirnov test can also be applied in this case to test whether
the data follows a uniform distribution.

\begin{longtable}[]{@{}cc@{}}
\caption{Theta Values with Non-Rejected KS Test}\tabularnewline
\toprule\noalign{}
Theta & P-Value \\
\midrule\noalign{}
\endfirsthead
\toprule\noalign{}
Theta & P-Value \\
\midrule\noalign{}
\endhead
\bottomrule\noalign{}
\endlastfoot
7.0 & 0.038 \\
7.1 & 0.053 \\
7.2 & 0.071 \\
7.3 & 0.092 \\
7.4 & 0.117 \\
7.5 & 0.146 \\
7.6 & 0.179 \\
7.7 & 0.216 \\
7.8 & 0.256 \\
7.9 & 0.299 \\
8.0 & 0.345 \\
8.1 & 0.394 \\
8.2 & 0.444 \\
8.3 & 0.495 \\
8.4 & 0.509 \\
8.5 & 0.419 \\
8.6 & 0.342 \\
8.7 & 0.277 \\
8.8 & 0.223 \\
8.9 & 0.179 \\
9.0 & 0.143 \\
9.1 & 0.114 \\
9.2 & 0.091 \\
9.3 & 0.072 \\
9.4 & 0.057 \\
9.5 & 0.045 \\
9.6 & 0.036 \\
\end{longtable}

\textbf{e)} Finally, we are testing the following Null hypothesis:
\emph{Null hypothesis (}\(H_0\)): The median cholesterol level after 8
weeks is 6. With the results presented below we can conclude there is
not enough statistical evidence to conclude that the median cholesterol
level after 8 weeks is less than 6. While 61.1\% of the sample is below
6, this deviation could easily be due to random variation given the
sample size (p\textgreater0.1).

\begin{longtable}[]{@{}ll@{}}
\caption{Median Test Results (H₀: median = 6)}\tabularnewline
\toprule\noalign{}
Statistic & Value \\
\midrule\noalign{}
\endfirsthead
\toprule\noalign{}
Statistic & Value \\
\midrule\noalign{}
\endhead
\bottomrule\noalign{}
\endlastfoot
Sample Size & 18 \\
Number \textless{} 6 & 11 \\
Observed Proportion & 0.611 \\
p-value & 0.24 \\
95\% CI & 0.392 to 1 \\
\end{longtable}

Subsequently, our second Null hypothesis goes as following: \emph{Null
hypothesis (}\(H_0\)): the fraction of cholesterol levels below 4.5 is
at most 0.25. Similarly, we also cannot reject this hypothesis because
of the very high p-value (p\textgreater0.1) and a wide CI.

\begin{longtable}[]{@{}ll@{}}
\caption{Fraction Test Results (H₀: fraction below 4.5 is
25\%)}\tabularnewline
\toprule\noalign{}
Statistic & Value \\
\midrule\noalign{}
\endfirsthead
\toprule\noalign{}
Statistic & Value \\
\midrule\noalign{}
\endhead
\bottomrule\noalign{}
\endlastfoot
Sample Size & 18 \\
Number \textless{} 4.5 & 3 \\
Observed Proportion & 0.167 \\
p-value & 0.865 \\
95\% CI & 0.047 to 1.000 \\
\end{longtable}

\subsection{Exercise 2}\label{exercise-2}

\paragraph{Section a}\label{section-a}

Before conducting the ANOVA test, we first plotted two interaction plots
to get a first glimpse of the potential interaction effect. Based on the
two interaction plots, it seems there is little interaction effect, as
the lines are parallel in general. We then conduct a two-way ANOVA to
confirm our observation.

\begin{Shaded}
\begin{Highlighting}[]
\FunctionTok{interaction.plot}\NormalTok{(related,county,crops) }\CommentTok{\# fix county}
\end{Highlighting}
\end{Shaded}

\includegraphics{assignment1_files/figure-latex/unnamed-chunk-15-1.pdf}

\begin{Shaded}
\begin{Highlighting}[]
\FunctionTok{interaction.plot}\NormalTok{(county,related,crops) }\CommentTok{\# fix related}
\end{Highlighting}
\end{Shaded}

\includegraphics{assignment1_files/figure-latex/unnamed-chunk-15-2.pdf}

\begin{Shaded}
\begin{Highlighting}[]
\FunctionTok{is.factor}\NormalTok{(county) }\CommentTok{\# check if county and related are factors}
\FunctionTok{is.factor}\NormalTok{(related)}
\end{Highlighting}
\end{Shaded}

\begin{Shaded}
\begin{Highlighting}[]
\NormalTok{cropanov}\OtherTok{=}\FunctionTok{lm}\NormalTok{(crops}\SpecialCharTok{\textasciitilde{}}\NormalTok{county}\SpecialCharTok{*}\NormalTok{related); }\FunctionTok{anova}\NormalTok{(cropanov) }\CommentTok{\# conduct the two{-}way ANOVA test}
\FunctionTok{summary}\NormalTok{(cropanov)}
\end{Highlighting}
\end{Shaded}

Result shows that there is no interaction effect between \emph{Related}
and \emph{County} on \emph{Crops}. None of the p values for
\emph{County}, \emph{Crops} and \emph{County:Related} are significant (p
= 0.477; p = 0.527; p = 0.879). To make sure that this result is valid,
we plot a Q-Q plot and residual plot. Based on the two plots, the
assumption of normality is met: Q-Q plot gives a straight line in
general, and the residuals display no pattern.

\begin{Shaded}
\begin{Highlighting}[]
\FunctionTok{par}\NormalTok{(}\AttributeTok{mfrow=}\FunctionTok{c}\NormalTok{(}\DecValTok{1}\NormalTok{,}\DecValTok{2}\NormalTok{))}
\FunctionTok{qqnorm}\NormalTok{(}\FunctionTok{residuals}\NormalTok{(cropanov)); }\FunctionTok{plot}\NormalTok{(}\FunctionTok{fitted}\NormalTok{(cropanov),}\FunctionTok{residuals}\NormalTok{(cropanov))}
\end{Highlighting}
\end{Shaded}

\includegraphics{assignment1_files/figure-latex/unnamed-chunk-18-1.pdf}

In the next step, we remove the interaction and apply an additive model.
The code and results are shown below:

\begin{Shaded}
\begin{Highlighting}[]
\NormalTok{cropanov2}\OtherTok{=}\FunctionTok{lm}\NormalTok{(crops}\SpecialCharTok{\textasciitilde{}}\NormalTok{county}\SpecialCharTok{+}\NormalTok{related,}\AttributeTok{data=}\NormalTok{cropframe); }\FunctionTok{anova}\NormalTok{(cropanov2) }\CommentTok{\# additive model}
\FunctionTok{summary}\NormalTok{(cropanov2)}
\end{Highlighting}
\end{Shaded}

The result of the additive model shows that neither of the factors has a
significant main effect on Crops. The p-values are 0.4518 and 0.5126 for
\emph{County} and \emph{Related} respectively, and are larger than the
0.05 significance level in both cases. Therefore, we fail to reject none
of our hypotheses. The normality assumption of this ANOVA test is also
met based on the following Q-Q plot and residual plot:

\begin{Shaded}
\begin{Highlighting}[]
\FunctionTok{qqnorm}\NormalTok{(}\FunctionTok{residuals}\NormalTok{(cropanov2)); }\FunctionTok{plot}\NormalTok{(}\FunctionTok{fitted}\NormalTok{(cropanov2),}\FunctionTok{residuals}\NormalTok{(cropanov2))}
\end{Highlighting}
\end{Shaded}

\includegraphics{assignment1_files/figure-latex/unnamed-chunk-20-1.pdf}
\includegraphics{assignment1_files/figure-latex/unnamed-chunk-20-2.pdf}

Summary for the decisions to the null hypotheses:

\begin{longtable}[]{@{}ll@{}}
\toprule\noalign{}
Hypothesis & Decision \\
\midrule\noalign{}
\endhead
\bottomrule\noalign{}
\endlastfoot
\(H_{AB}\) & not reject \\
\(H_{A}\) & not reject \\
\(H_{B}\) & not reject \\
\end{longtable}

The mathematical formula for a two-way ANOVA model is:

\[
Y_{ijk} = \mu_{ij} + e_{ijk}
\]

where

\[
\mu_{ij} = \mu + \alpha_i + \beta_j + \gamma_{ij}
\]

\(\mu\) is the overall mean

\(\alpha_i\) is the main effect of level i of the factor \emph{County},
i = 1,2,3

\(\beta_j\) is the main effect of level j of the factor \emph{Related},
j = 0,1

\(\gamma_ij\) is the interaction effect of levels i, j of factor
\emph{County} and \emph{Related}, which is 0 in this case, since there
is no significant interaction effect.

We apply the model \textbf{cropanov2} for prediction.Therefore, the
crops in \emph{County 3} for which there is no related is:

\[
Crops = Intercept + County3 + Related0 = 6800.6 + 959.7 + 0 = 7760.3
\]

Therefore, the predicted value of the Crops is 7760.3.

\paragraph{Section b}\label{section-b}

\begin{Shaded}
\begin{Highlighting}[]
\NormalTok{size }\OtherTok{=}\NormalTok{ crop\_data}\SpecialCharTok{$}\NormalTok{Size}
\FunctionTok{boxplot}\NormalTok{(size}\SpecialCharTok{\textasciitilde{}}\NormalTok{county)}
\end{Highlighting}
\end{Shaded}

\includegraphics{assignment1_files/figure-latex/unnamed-chunk-21-1.pdf}

\begin{Shaded}
\begin{Highlighting}[]
\NormalTok{cropanov3}\OtherTok{=}\FunctionTok{lm}\NormalTok{(crops}\SpecialCharTok{\textasciitilde{}}\NormalTok{size}\SpecialCharTok{*}\NormalTok{county,}\AttributeTok{data=}\NormalTok{cropframe);}\FunctionTok{anova}\NormalTok{(cropanov3)}
\FunctionTok{summary}\NormalTok{(cropanov3)}
\end{Highlighting}
\end{Shaded}

Based on the result, there is a significant interaction effect between
\emph{Size} and \emph{County} on \emph{Crops (p-value = 0.007)}. Summary
of the ANOVA model shows that the effect mainly lies on the combination
of \emph{size:county 2 (p-value = 0.002)}, while size:county 3 is not
significant (p-value = 0.157). Meanwhile, \emph{county 2} also has a
significant main effect under the influence of \emph{Size} (p-value =
0.005). Q-Q plot and residual plot show that the assumption of normality
is met in this case:

\begin{Shaded}
\begin{Highlighting}[]
\FunctionTok{qqnorm}\NormalTok{(}\FunctionTok{residuals}\NormalTok{(cropanov3)); }\FunctionTok{plot}\NormalTok{(}\FunctionTok{fitted}\NormalTok{(cropanov3),}\FunctionTok{residuals}\NormalTok{(cropanov3))}
\end{Highlighting}
\end{Shaded}

\includegraphics{assignment1_files/figure-latex/unnamed-chunk-23-1.pdf}
\includegraphics{assignment1_files/figure-latex/unnamed-chunk-23-2.pdf}

We \textbf{reject the null hypothesis} that there is no interaction
effect between \emph{Size} and \emph{County}

\begin{enumerate}
\def\labelenumi{\arabic{enumi}.}
\setcounter{enumi}{1}
\tightlist
\item
  ANCOVA test: \emph{Size * Related}
\end{enumerate}

\(H_{AB}\) There is no interaction effect between \emph{Size} and
\emph{Related} on \emph{Crops}

\begin{Shaded}
\begin{Highlighting}[]
\FunctionTok{boxplot}\NormalTok{(size}\SpecialCharTok{\textasciitilde{}}\NormalTok{related)}
\end{Highlighting}
\end{Shaded}

\includegraphics{assignment1_files/figure-latex/unnamed-chunk-24-1.pdf}

\begin{Shaded}
\begin{Highlighting}[]
\NormalTok{cropanov4}\OtherTok{=}\FunctionTok{lm}\NormalTok{(crops}\SpecialCharTok{\textasciitilde{}}\NormalTok{size}\SpecialCharTok{*}\NormalTok{related,}\AttributeTok{data=}\NormalTok{cropframe);}\FunctionTok{anova}\NormalTok{(cropanov4)}
\FunctionTok{summary}\NormalTok{(cropanov4)}
\end{Highlighting}
\end{Shaded}

In the box plot, the distribution of \emph{Size} does not differ much
for different \emph{Related} values. The interaction effect is also not
significant according to the result of the ANCOVA test (p=0.331).
Therefore, we \textbf{cannot reject the null hypothesis} in this case.
The assumption of normality is met in general, though one can argue that
there are more residuals on the lower fitted value side.

\begin{Shaded}
\begin{Highlighting}[]
\FunctionTok{qqnorm}\NormalTok{(}\FunctionTok{residuals}\NormalTok{(cropanov4)); }\FunctionTok{plot}\NormalTok{(}\FunctionTok{fitted}\NormalTok{(cropanov4),}\FunctionTok{residuals}\NormalTok{(cropanov4))}
\end{Highlighting}
\end{Shaded}

\includegraphics{assignment1_files/figure-latex/unnamed-chunk-26-1.pdf}
\includegraphics{assignment1_files/figure-latex/unnamed-chunk-26-2.pdf}

We then conduct two ANCOVA tests without interaction. We investigate the
main effect of \emph{Related} under the influence of \emph{Size
(cropanov5)}, and the main effect of \emph{Size} under the influence of
\emph{Related (cropanov6).}

\begin{Shaded}
\begin{Highlighting}[]
\NormalTok{cropanov5}\OtherTok{=}\FunctionTok{lm}\NormalTok{(crops}\SpecialCharTok{\textasciitilde{}}\NormalTok{size}\SpecialCharTok{+}\NormalTok{related,}\AttributeTok{data=}\NormalTok{cropframe);}\FunctionTok{anova}\NormalTok{(cropanov5)}\CommentTok{\# related on the second place}
\FunctionTok{summary}\NormalTok{(cropanov5)}
\end{Highlighting}
\end{Shaded}

\begin{Shaded}
\begin{Highlighting}[]
\NormalTok{cropanov6}\OtherTok{=}\FunctionTok{lm}\NormalTok{(crops}\SpecialCharTok{\textasciitilde{}}\NormalTok{related}\SpecialCharTok{+}\NormalTok{size,}\AttributeTok{data=}\NormalTok{cropframe);}\FunctionTok{anova}\NormalTok{(cropanov6)}\CommentTok{\# size on the second place}
\FunctionTok{summary}\NormalTok{(cropanov6)}
\end{Highlighting}
\end{Shaded}

Results show that \emph{Related} does not have a significant main effect
on \emph{Crops} under the influence of \emph{Size}, but \emph{Size} has
a significant main effect under the influence of \emph{Related}. The
normality assumption of both of the ANCOVA tests are met in general,
although there seems to be more residuals in the area with a lower
fitted score:

\begin{Shaded}
\begin{Highlighting}[]
\FunctionTok{qqnorm}\NormalTok{(}\FunctionTok{residuals}\NormalTok{(cropanov5)); }\FunctionTok{plot}\NormalTok{(}\FunctionTok{fitted}\NormalTok{(cropanov5),}\FunctionTok{residuals}\NormalTok{(cropanov5))}
\end{Highlighting}
\end{Shaded}

\includegraphics{assignment1_files/figure-latex/unnamed-chunk-29-1.pdf}
\includegraphics{assignment1_files/figure-latex/unnamed-chunk-29-2.pdf}

\begin{Shaded}
\begin{Highlighting}[]
\FunctionTok{qqnorm}\NormalTok{(}\FunctionTok{residuals}\NormalTok{(cropanov6)); }\FunctionTok{plot}\NormalTok{(}\FunctionTok{fitted}\NormalTok{(cropanov4),}\FunctionTok{residuals}\NormalTok{(cropanov6))}
\end{Highlighting}
\end{Shaded}

\includegraphics{assignment1_files/figure-latex/unnamed-chunk-30-1.pdf}
\includegraphics{assignment1_files/figure-latex/unnamed-chunk-30-2.pdf}

Summary of this part:

\begin{enumerate}
\def\labelenumi{(\arabic{enumi})}
\item
  There is a significant interaction effect between \emph{Size} and
  \emph{County} on \emph{Crops}. The combination \emph{Size:County 2} is
  making the most contribution.
\item
  No significant interaction effect is found for \emph{Size * Related}.
\item
  \emph{Related} has no significant main effect on \emph{Crops} under
  the influence of \emph{Size}, while \emph{Size} has a significant main
  effect on \emph{Crops} under the influence of \emph{Related}.
\end{enumerate}

\paragraph{Section c}\label{section-c}

\begin{Shaded}
\begin{Highlighting}[]
\NormalTok{cropanov7}\OtherTok{=}\FunctionTok{lm}\NormalTok{(crops }\SpecialCharTok{\textasciitilde{}}\NormalTok{ county}\SpecialCharTok{+}\NormalTok{related}\SpecialCharTok{+}\NormalTok{size}\SpecialCharTok{+}\NormalTok{county}\SpecialCharTok{*}\NormalTok{size, }\AttributeTok{data =}\NormalTok{ crop\_data);}\FunctionTok{anova}\NormalTok{(cropanov7)}\CommentTok{\# size on the second place}
\end{Highlighting}
\end{Shaded}

\begin{verbatim}
## Analysis of Variance Table
## 
## Response: crops
##             Df   Sum Sq  Mean Sq F value  Pr(>F)    
## county       2 8.84e+06 4.42e+06    5.01   0.016 *  
## related      1 2.38e+06 2.38e+06    2.70   0.114    
## size         1 1.10e+08 1.10e+08  125.33 8.7e-11 ***
## county:size  2 9.53e+06 4.76e+06    5.40   0.012 *  
## Residuals   23 2.03e+07 8.82e+05                    
## ---
## Signif. codes:  0 '***' 0.001 '**' 0.01 '*' 0.05 '.' 0.1 ' ' 1
\end{verbatim}

\begin{Shaded}
\begin{Highlighting}[]
\FunctionTok{summary}\NormalTok{(cropanov7)}
\end{Highlighting}
\end{Shaded}

\begin{verbatim}
## 
## Call:
## lm(formula = crops ~ county + related + size + county * size, 
##     data = crop_data)
## 
## Residuals:
##     Min      1Q  Median      3Q     Max 
## -1477.6  -517.1    63.9   639.6  1690.3 
## 
## Coefficients:
##              Estimate Std. Error t value Pr(>|t|)    
## (Intercept)   2461.01     929.76    2.65   0.0144 *  
## county2      -4214.05    1447.24   -2.91   0.0079 ** 
## county3      -1284.81    1302.58   -0.99   0.3342    
## related1      -239.10     347.92   -0.69   0.4988    
## size            22.70       4.77    4.76  8.4e-05 ***
## county2:size    26.59       8.09    3.29   0.0032 ** 
## county3:size     8.92       6.40    1.39   0.1768    
## ---
## Signif. codes:  0 '***' 0.001 '**' 0.01 '*' 0.05 '.' 0.1 ' ' 1
## 
## Residual standard error: 939 on 23 degrees of freedom
## Multiple R-squared:  0.866,  Adjusted R-squared:  0.831 
## F-statistic: 24.8 on 6 and 23 DF,  p-value: 5.85e-09
\end{verbatim}

\begin{Shaded}
\begin{Highlighting}[]
\FunctionTok{qqnorm}\NormalTok{(}\FunctionTok{residuals}\NormalTok{(cropanov7)); }\FunctionTok{plot}\NormalTok{(}\FunctionTok{fitted}\NormalTok{(cropanov7),}\FunctionTok{residuals}\NormalTok{(cropanov7))}
\end{Highlighting}
\end{Shaded}

\includegraphics{assignment1_files/figure-latex/unnamed-chunk-32-1.pdf}
\includegraphics{assignment1_files/figure-latex/unnamed-chunk-32-2.pdf}

Result of a full ANCOVA test also confirms what we found in section B in
general. There is a significant main effect of variable \emph{Size (p =
0.000)} and \emph{County 2 (p = 0.008)}. The interaction effect of
\emph{Size:County} is therefore also significant \emph{(p = 0.012)}.
Different from section B, \emph{County} now also has a slightly
significant main effect \emph{(p = 0.016)}, when \emph{County, Related}
and \emph{Size} are all included into the model. The assumption of
normality is met according to the Q-Q plot and residual plot.

\paragraph{Section d}\label{section-d}

We will apply model \textbf{cropanov7} to make the prediction. The
mathematical formula for a full ANCOVA is:

\[ Y_{ijk} = \mu_{ij} + e_{ijk} \]

According to this equation, the crops from County 2 of size 165, and
related landlord and tenant is therefore:

\[
Crops = Intercept + County2 + Related1 + Size 165 + County2*Size165
\]

So the final crops value is 6141.378

The error variance is given by:

\[
\hat{\sigma}^2 = \frac{\text{RSS}}{\text{df}}
\]

According to the summary of the \textbf{cropanov7,} we then have:\[
\hat{\sigma}^2 = \frac{20277325}{23} = 881623
\]

The error variance is therefore 881623

\subsection{Exercise 3: Yield of peas}\label{exercise-3-yield-of-peas}

\subsubsection{Section a}\label{section-a-1}

\begin{Shaded}
\begin{Highlighting}[]
\FunctionTok{library}\NormalTok{(}\StringTok{\textquotesingle{}MASS\textquotesingle{}}\NormalTok{)}

\FunctionTok{set.seed}\NormalTok{(}\DecValTok{123}\NormalTok{)  }\CommentTok{\# add random seed for reproduce}

\CommentTok{\# initial params}
\NormalTok{I }\OtherTok{\textless{}{-}} \DecValTok{6}  \CommentTok{\# blocks}
\NormalTok{J }\OtherTok{\textless{}{-}} \DecValTok{4}  \CommentTok{\# plots per block}

\CommentTok{\# initial data frame}
\NormalTok{randomized\_design }\OtherTok{\textless{}{-}} \FunctionTok{data.frame}\NormalTok{(}\AttributeTok{block =} \FunctionTok{rep}\NormalTok{(}\DecValTok{1}\SpecialCharTok{:}\NormalTok{I, }\AttributeTok{each =}\NormalTok{ J), }\AttributeTok{plot =} \FunctionTok{rep}\NormalTok{(}\DecValTok{1}\SpecialCharTok{:}\NormalTok{J, }\AttributeTok{times =}\NormalTok{ I))}

\CommentTok{\# for each block b, put (N, P, K) on each 2 plots randomly}
\ControlFlowTok{for}\NormalTok{ (b }\ControlFlowTok{in} \DecValTok{1}\SpecialCharTok{:}\NormalTok{I) \{}
\NormalTok{  plots }\OtherTok{\textless{}{-}} \FunctionTok{sample}\NormalTok{(}\DecValTok{1}\SpecialCharTok{:}\NormalTok{J, J, }\AttributeTok{replace =} \ConstantTok{FALSE}\NormalTok{)  }\CommentTok{\# randomly reorder plots in each block}
  
  \CommentTok{\# put N in the header 2 plots}
\NormalTok{  randomized\_design}\SpecialCharTok{$}\NormalTok{N[randomized\_design}\SpecialCharTok{$}\NormalTok{block }\SpecialCharTok{==}\NormalTok{ b] }\OtherTok{\textless{}{-}} \FunctionTok{ifelse}\NormalTok{(plots }\SpecialCharTok{\%in\%}\NormalTok{ plots[}\DecValTok{1}\SpecialCharTok{:}\DecValTok{2}\NormalTok{], }\DecValTok{1}\NormalTok{, }\DecValTok{0}\NormalTok{)}
  
  \CommentTok{\# randomly put P in 2 plots}
\NormalTok{  randomized\_design}\SpecialCharTok{$}\NormalTok{P[randomized\_design}\SpecialCharTok{$}\NormalTok{block }\SpecialCharTok{==}\NormalTok{ b] }\OtherTok{\textless{}{-}} \FunctionTok{ifelse}\NormalTok{(plots }\SpecialCharTok{\%in\%} \FunctionTok{sample}\NormalTok{(plots, }\DecValTok{2}\NormalTok{), }\DecValTok{1}\NormalTok{, }\DecValTok{0}\NormalTok{)}
  
  \CommentTok{\# randomly put K in 2 plots}
\NormalTok{  randomized\_design}\SpecialCharTok{$}\NormalTok{K[randomized\_design}\SpecialCharTok{$}\NormalTok{block }\SpecialCharTok{==}\NormalTok{ b] }\OtherTok{\textless{}{-}} \FunctionTok{ifelse}\NormalTok{(plots }\SpecialCharTok{\%in\%} \FunctionTok{sample}\NormalTok{(plots, }\DecValTok{2}\NormalTok{), }\DecValTok{1}\NormalTok{, }\DecValTok{0}\NormalTok{)}
\NormalTok{\}}

\CommentTok{\# print the plots}
\FunctionTok{print}\NormalTok{(randomized\_design)}
\end{Highlighting}
\end{Shaded}

\begin{verbatim}
##    block plot N P K
## 1      1    1 1 0 0
## 2      1    2 1 1 1
## 3      1    3 0 1 0
## 4      1    4 0 0 1
## 5      2    1 1 1 0
## 6      2    2 1 1 1
## 7      2    3 0 0 1
## 8      2    4 0 0 0
## 9      3    1 1 1 1
## 10     3    2 1 0 0
## 11     3    3 0 0 0
## 12     3    4 0 1 1
## 13     4    1 1 1 0
## 14     4    2 1 1 1
## 15     4    3 0 0 1
## 16     4    4 0 0 0
## 17     5    1 1 0 0
## 18     5    2 1 0 1
## 19     5    3 0 1 0
## 20     5    4 0 1 1
## 21     6    1 1 0 1
## 22     6    2 1 1 0
## 23     6    3 0 1 1
## 24     6    4 0 0 0
\end{verbatim}

\subsubsection{Section b}\label{section-b-1}

\begin{Shaded}
\begin{Highlighting}[]
\CommentTok{\# combine those yield in the same block same N, and calc its mean}
\NormalTok{yield\_matrix }\OtherTok{\textless{}{-}} \FunctionTok{tapply}\NormalTok{(npk}\SpecialCharTok{$}\NormalTok{yield, }\FunctionTok{list}\NormalTok{(npk}\SpecialCharTok{$}\NormalTok{block, npk}\SpecialCharTok{$}\NormalTok{N), mean)}

\CommentTok{\# plot bars}
\FunctionTok{barplot}\NormalTok{(}\FunctionTok{t}\NormalTok{(yield\_matrix), }\AttributeTok{beside =} \ConstantTok{TRUE}\NormalTok{, }\AttributeTok{col =} \FunctionTok{c}\NormalTok{(}\StringTok{"red"}\NormalTok{, }\StringTok{"blue"}\NormalTok{),}
        \AttributeTok{main =} \StringTok{"Average Yield per Block for Nitrogen Treatment"}\NormalTok{,}
        \AttributeTok{xlab =} \StringTok{"Block"}\NormalTok{, }\AttributeTok{ylab =} \StringTok{"Average Yield"}\NormalTok{)}
\FunctionTok{legend}\NormalTok{(}\StringTok{"top"}\NormalTok{, }\AttributeTok{legend =} \FunctionTok{c}\NormalTok{(}\StringTok{"N=0"}\NormalTok{, }\StringTok{"N=1"}\NormalTok{), }\AttributeTok{fill =} \FunctionTok{c}\NormalTok{(}\StringTok{"red"}\NormalTok{, }\StringTok{"blue"}\NormalTok{))}
\end{Highlighting}
\end{Shaded}

\includegraphics{assignment1_files/figure-latex/unnamed-chunk-34-1.pdf}

This plot illustrates that the average yields for soil treated by N are
higher than for untreated soil. What's more, each block and treatment
tend to have a similar change.

Meanwhile, we here have assigned treatments randomly to each soil within
a block, which reduces the variation and get more precise results.

\subsubsection{Section c}\label{section-c-1}

\begin{Shaded}
\begin{Highlighting}[]
\NormalTok{data0 }\OtherTok{=}\NormalTok{ npk}
\NormalTok{data0}\SpecialCharTok{$}\NormalTok{block }\OtherTok{=} \FunctionTok{as.factor}\NormalTok{(data0}\SpecialCharTok{$}\NormalTok{block)}
\NormalTok{data0}\SpecialCharTok{$}\NormalTok{N }\OtherTok{=} \FunctionTok{as.factor}\NormalTok{(data0}\SpecialCharTok{$}\NormalTok{N)}
\end{Highlighting}
\end{Shaded}

\begin{Shaded}
\begin{Highlighting}[]
\CommentTok{\# Two{-}Way ANOVA}
\NormalTok{model2way }\OtherTok{=} \FunctionTok{lm}\NormalTok{(yield}\SpecialCharTok{\textasciitilde{}}\NormalTok{N}\SpecialCharTok{*}\NormalTok{block, }\AttributeTok{data=}\NormalTok{data0)}
\FunctionTok{anova}\NormalTok{(model2way)}
\end{Highlighting}
\end{Shaded}

\begin{verbatim}
## Analysis of Variance Table
## 
## Response: yield
##           Df Sum Sq Mean Sq F value Pr(>F)  
## N          1    189   189.3    9.26   0.01 *
## block      5    343    68.7    3.36   0.04 *
## N:block    5     99    19.7    0.96   0.48  
## Residuals 12    245    20.4                 
## ---
## Signif. codes:  0 '***' 0.001 '**' 0.01 '*' 0.05 '.' 0.1 ' ' 1
\end{verbatim}

p \textgreater{} 0.05, which means there is no significant evidence of
interaction effect.

\begin{Shaded}
\begin{Highlighting}[]
\FunctionTok{interaction.plot}\NormalTok{(data0}\SpecialCharTok{$}\NormalTok{N,data0}\SpecialCharTok{$}\NormalTok{block,data0}\SpecialCharTok{$}\NormalTok{yield)}
\end{Highlighting}
\end{Shaded}

\includegraphics{assignment1_files/figure-latex/unnamed-chunk-37-1.pdf}

\begin{Shaded}
\begin{Highlighting}[]
\FunctionTok{interaction.plot}\NormalTok{(data0}\SpecialCharTok{$}\NormalTok{block,data0}\SpecialCharTok{$}\NormalTok{N,data0}\SpecialCharTok{$}\NormalTok{yield)}
\end{Highlighting}
\end{Shaded}

\includegraphics{assignment1_files/figure-latex/unnamed-chunk-37-2.pdf}

Interaction plot also display parallel lines, indicating no interaction.

So, we have to try the ``additive'' model:

\begin{Shaded}
\begin{Highlighting}[]
\NormalTok{modeladd }\OtherTok{\textless{}{-}} \FunctionTok{lm}\NormalTok{(yield }\SpecialCharTok{\textasciitilde{}}\NormalTok{ N }\SpecialCharTok{+}\NormalTok{ block, }\AttributeTok{data =}\NormalTok{ data0)}
\FunctionTok{anova}\NormalTok{(modeladd)}
\end{Highlighting}
\end{Shaded}

\begin{verbatim}
## Analysis of Variance Table
## 
## Response: yield
##           Df Sum Sq Mean Sq F value Pr(>F)   
## N          1    189   189.3    9.36 0.0071 **
## block      5    343    68.7    3.40 0.0262 * 
## Residuals 17    344    20.2                  
## ---
## Signif. codes:  0 '***' 0.001 '**' 0.01 '*' 0.05 '.' 0.1 ' ' 1
\end{verbatim}

In both cases p \textless{} 0.05, so both factors have a main effect in
the ``additive'' model.

\begin{Shaded}
\begin{Highlighting}[]
\CommentTok{\# Diagnostics:}
\FunctionTok{par}\NormalTok{(}\AttributeTok{mfrow=}\FunctionTok{c}\NormalTok{(}\DecValTok{1}\NormalTok{,}\DecValTok{2}\NormalTok{)) }
\FunctionTok{qqnorm}\NormalTok{(}\FunctionTok{residuals}\NormalTok{(modeladd)); }\FunctionTok{plot}\NormalTok{(}\FunctionTok{fitted}\NormalTok{(modeladd),}\FunctionTok{residuals}\NormalTok{(modeladd))}
\end{Highlighting}
\end{Shaded}

\includegraphics{assignment1_files/figure-latex/unnamed-chunk-39-1.pdf}

From QQPlot, we can tell that the curve more or less straight, so it is
likely normal. Meanwhile, there is no significant pattern in the fitted
plot, which is good and means the residual is independent and identical.

\begin{itemize}
\tightlist
\item
  \textbf{Was it sensible to include factor block into this model?} From
  the results showed in ``additive'' model, the p\_value of block is
  0.007095 \textless{} 0.05, and the N is the first order in our model,
  so it makes sense to include the block.
\item
  \textbf{Can we also apply the Friedman test for this situation?} No,
  because each block has more than one same value N, meanwhile, the
  treatments are not completely randomized.
\end{itemize}

\subsubsection{Section d}\label{section-d-1}

\begin{Shaded}
\begin{Highlighting}[]
\NormalTok{pairwiseP }\OtherTok{\textless{}{-}} \FunctionTok{lm}\NormalTok{(yield }\SpecialCharTok{\textasciitilde{}}\NormalTok{ block}\SpecialCharTok{*}\NormalTok{P }\SpecialCharTok{+}\NormalTok{ N }\SpecialCharTok{+}\NormalTok{ K, }\AttributeTok{data =}\NormalTok{ data0) }
\NormalTok{pairwiseK }\OtherTok{\textless{}{-}} \FunctionTok{lm}\NormalTok{(yield }\SpecialCharTok{\textasciitilde{}}\NormalTok{ block}\SpecialCharTok{*}\NormalTok{K }\SpecialCharTok{+}\NormalTok{ P }\SpecialCharTok{+}\NormalTok{ N, }\AttributeTok{data =}\NormalTok{ data0) }
\NormalTok{pairwiseN }\OtherTok{\textless{}{-}} \FunctionTok{lm}\NormalTok{(yield }\SpecialCharTok{\textasciitilde{}}\NormalTok{ block}\SpecialCharTok{*}\NormalTok{N }\SpecialCharTok{+}\NormalTok{ K }\SpecialCharTok{+}\NormalTok{ P, }\AttributeTok{data =}\NormalTok{ data0)}
\end{Highlighting}
\end{Shaded}

\begin{Shaded}
\begin{Highlighting}[]
\FunctionTok{anova}\NormalTok{(pairwiseP); }\FunctionTok{anova}\NormalTok{(pairwiseK); }\FunctionTok{anova}\NormalTok{(pairwiseN)}
\end{Highlighting}
\end{Shaded}

\begin{verbatim}
## Analysis of Variance Table
## 
## Response: yield
##           Df Sum Sq Mean Sq F value Pr(>F)   
## block      5    343    68.7    4.07 0.0282 * 
## P          1      8     8.4    0.50 0.4966   
## N          1    189   189.3   11.21 0.0074 **
## K          1     95    95.2    5.64 0.0389 * 
## block:P    5     71    14.3    0.85 0.5473   
## Residuals 10    169    16.9                  
## ---
## Signif. codes:  0 '***' 0.001 '**' 0.01 '*' 0.05 '.' 0.1 ' ' 1
\end{verbatim}

\begin{verbatim}
## Analysis of Variance Table
## 
## Response: yield
##           Df Sum Sq Mean Sq F value Pr(>F)   
## block      5    343    68.7    4.04 0.0288 * 
## K          1     95    95.2    5.60 0.0395 * 
## P          1      8     8.4    0.49 0.4980   
## N          1    189   189.3   11.14 0.0075 **
## block:K    5     70    14.1    0.83 0.5583   
## Residuals 10    170    17.0                  
## ---
## Signif. codes:  0 '***' 0.001 '**' 0.01 '*' 0.05 '.' 0.1 ' ' 1
\end{verbatim}

\begin{verbatim}
## Analysis of Variance Table
## 
## Response: yield
##           Df Sum Sq Mean Sq F value Pr(>F)   
## block      5    343    68.7    4.85 0.0164 * 
## N          1    189   189.3   13.36 0.0044 **
## K          1     95    95.2    6.72 0.0268 * 
## P          1      8     8.4    0.59 0.4590   
## block:N    5     99    19.7    1.39 0.3066   
## Residuals 10    142    14.2                  
## ---
## Signif. codes:  0 '***' 0.001 '**' 0.01 '*' 0.05 '.' 0.1 ' ' 1
\end{verbatim}

No interaction effect for either of the three, thus we do an additive
model:

\begin{Shaded}
\begin{Highlighting}[]
\NormalTok{modeladd2 }\OtherTok{\textless{}{-}} \FunctionTok{lm}\NormalTok{(yield }\SpecialCharTok{\textasciitilde{}}\NormalTok{ block }\SpecialCharTok{+}\NormalTok{ N }\SpecialCharTok{+}\NormalTok{ P }\SpecialCharTok{+}\NormalTok{ K, }\AttributeTok{data =}\NormalTok{ data0); }\FunctionTok{anova}\NormalTok{(modeladd2)}
\end{Highlighting}
\end{Shaded}

\begin{verbatim}
## Analysis of Variance Table
## 
## Response: yield
##           Df Sum Sq Mean Sq F value Pr(>F)   
## block      5    343    68.7    4.29 0.0127 * 
## N          1    189   189.3   11.82 0.0037 **
## P          1      8     8.4    0.52 0.4800   
## K          1     95    95.2    5.95 0.0277 * 
## Residuals 15    240    16.0                  
## ---
## Signif. codes:  0 '***' 0.001 '**' 0.01 '*' 0.05 '.' 0.1 ' ' 1
\end{verbatim}

p \textless{} 0.05 for N and K, showing a main effect. p \textgreater{}
0.05 for P, so we can conclude that there is no significant effect.

\textbf{We conclude that the best model is the additive ANOVA model.}
Additive ANOVA model provides an overall indication of the effects of
each factor N, P, K and block. From the p values we can tell additive
model are better than the pairwise model. Furthermore, the pairwise
models only focus on the interaction between two factors at a time,
lacking the control for others.

\subsubsection{Section e}\label{section-e}

\begin{Shaded}
\begin{Highlighting}[]
\NormalTok{model\_d }\OtherTok{=} \FunctionTok{lm}\NormalTok{(yield }\SpecialCharTok{\textasciitilde{}}\NormalTok{ block }\SpecialCharTok{+}\NormalTok{ N }\SpecialCharTok{+}\NormalTok{ K }\SpecialCharTok{+}\NormalTok{ P, }\AttributeTok{data =}\NormalTok{ data0)}
\FunctionTok{summary}\NormalTok{(model\_d)}
\end{Highlighting}
\end{Shaded}

\begin{verbatim}
## 
## Call:
## lm(formula = yield ~ block + N + K + P, data = data0)
## 
## Residuals:
##    Min     1Q Median     3Q    Max 
## -7.000 -1.708 -0.083  2.246  6.483 
## 
## Coefficients:
##             Estimate Std. Error t value Pr(>|t|)    
## (Intercept)    53.80       2.45   21.96  8.1e-13 ***
## block2          3.43       2.83    1.21   0.2448    
## block3          6.75       2.83    2.39   0.0307 *  
## block4         -3.90       2.83   -1.38   0.1883    
## block5         -3.50       2.83   -1.24   0.2351    
## block6          2.32       2.83    0.82   0.4241    
## N1              5.62       1.63    3.44   0.0037 ** 
## K1             -3.98       1.63   -2.44   0.0277 *  
## P1             -1.18       1.63   -0.72   0.4800    
## ---
## Signif. codes:  0 '***' 0.001 '**' 0.01 '*' 0.05 '.' 0.1 ' ' 1
## 
## Residual standard error: 4 on 15 degrees of freedom
## Multiple R-squared:  0.726,  Adjusted R-squared:  0.58 
## F-statistic: 4.97 on 8 and 15 DF,  p-value: 0.00376
\end{verbatim}

From the summary we can see block 3 is the best in all blocks, and N1 is
better than N0, which means N treated is better, while P and K are
prefered to be untreated. So, the best combination is \textbf{(3, 1, 0,
0)} for (block, N, P, K), leading the largest yield.

\subsubsection{Section f}\label{section-f}

\begin{Shaded}
\begin{Highlighting}[]
\FunctionTok{library}\NormalTok{(}\StringTok{\textquotesingle{}MASS\textquotesingle{}}\NormalTok{)}
\FunctionTok{library}\NormalTok{(}\StringTok{\textquotesingle{}lme4\textquotesingle{}}\NormalTok{)}
\NormalTok{data0 }\OtherTok{=}\NormalTok{ npk}

\NormalTok{model\_mixed }\OtherTok{\textless{}{-}} \FunctionTok{lmer}\NormalTok{(yield }\SpecialCharTok{\textasciitilde{}}\NormalTok{ N}\SpecialCharTok{+}\NormalTok{P}\SpecialCharTok{+}\NormalTok{K}\SpecialCharTok{+}\NormalTok{(}\DecValTok{1}\SpecialCharTok{|}\NormalTok{block),}\AttributeTok{data=}\NormalTok{data0,}\AttributeTok{REML=}\ConstantTok{FALSE}\NormalTok{) }
\NormalTok{model\_fixed }\OtherTok{\textless{}{-}} \FunctionTok{lm}\NormalTok{(yield }\SpecialCharTok{\textasciitilde{}}\NormalTok{  N }\SpecialCharTok{+}\NormalTok{ P }\SpecialCharTok{+}\NormalTok{ K }\SpecialCharTok{+}\NormalTok{ block, }\AttributeTok{data =}\NormalTok{ data0)}
\FunctionTok{anova}\NormalTok{(model\_mixed,model\_fixed) }
\end{Highlighting}
\end{Shaded}

\begin{verbatim}
## Data: data0
## Models:
## model_mixed: yield ~ N + P + K + (1 | block)
## model_fixed: yield ~ N + P + K + block
##             npar AIC BIC logLik deviance Chisq Df Pr(>Chisq)   
## model_mixed    6 151 158  -69.5      139                       
## model_fixed   10 143 155  -61.7      123  15.6  4     0.0035 **
## ---
## Signif. codes:  0 '***' 0.001 '**' 0.01 '*' 0.05 '.' 0.1 ' ' 1
\end{verbatim}

The model comparison results are:

\begin{itemize}
\tightlist
\item
  AIC: The fixed effects model (AIC = 143.39) is lower than the mixed
  effects model (AIC = 151.03), suggesting better model fit.
\item
  BIC: The fixed effects model (BIC = 155.17) is also lower, reinforcing
  the AIC results.
\item
  Log-likelihood: The fixed effects model has a higher log-likelihood
  (-61.695 vs.~-69.514), meaning it fits the data better.
\item
  Chi-square test: χ² = 15.639, p = 0.003544 (significant at p
  \textless{} 0.05), indicating that treating block as a fixed effect is
  more appropriate.
\end{itemize}

The fixed effects model (lm(yield \textasciitilde{} block + N + P + K))
provides a better fit than the mixed effects model.

\end{document}
